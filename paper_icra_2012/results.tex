\section{Results} \label{sec:results} 

With the captured data, it was possible to fill 72.9\% of the original
neighborhoods with, at least, a threshold of 500 points. The distribution of
number of points across the $X$ space is illustrated by Figure
\ref{fig:numberOfPointsPerBin}.

We performed a quantile-quantile regression with the data shown in the
histogram from Figure \ref{fig:distancesToPlane} with a gaussian distribution,
and obtained a coefficient of determination of 0.805, which validates the
hypothesis that the error $e_i$ can be approximated by a Gaussian random
variable with zero mean.  Figure \ref{fig:sampleMeanFigure} shows the average
of the error calculated from the samples at each neighborhood $N_j$. Since most
of its values are dispersed close to zero, we can say that the approximation of
a zero-centered distribution for each neighborhood does not compromise our
model, whilst making it simpler for real applications.

Next we will show the results of our model generation. In Figure \ref{fig:stdImageZeroMean} we can see
the variance $\sigma_j$ in each neighborhood $N_j$ and observe that $\sigma_j$
increases with distance, which is a trend we expected. One important
observation is that, according to our model, the angle $\alpha$ does not play a
major role in determining the variation $\sigma_j$.  In Figure \ref{fig:scatterWithZeroMean} we can see a
scatter plot showing our generated polynomial surface $\sigma(\alpha,d)$;
$\sigma_j$ obtained from our data set; and $\sigma_j$ obtained from our
validation set. The surface represents the best fit polynomial surface through
the original estimations. The green dots represent the original estimations;
the red ones were acquired from the XtionPRO, while the black points were
computed from a different Kinect sensor.

We asses the ability of our model to describe our data by the coefficient of
determination. This allows us to determine the percentage of points whose
variation are described by $\sigma(\alpha,d)$. In Table \ref{tab:CD_coefs} we
can see the value of this coefficient for our large data set and also our
validation set. Our results show that our model could be used for other
Kinect device with a reasonable confidence, but that a new model would be needed
for the Xtion, should the application require a high accuracy model. 

%{\setlength\abovedisplayskip{-4pt} \setlength\belowdisplayskip{-6pt} % needed to make spacing more compact
\begin{table}[h!]
\caption{Coefficient of determination.}
\begin{center}
\begin{tabular}{|c|c|c|}
\hline
Original Kinect & Another Kinect & ASUS XtionPRO \\ \hline
$90.0\%\ $ & $74.7\%\ $ & $50.7\%\ $ \\ \hline
\end{tabular}
\end{center}
\label{tab:CD_coefs}
\end{table}
%}

The resulting measurement noise model from our analysis is given by (\ref{eq:model}) and the list of coefficients are in Table \ref{tab:Poly_coefs}.

{\setlength\abovedisplayskip{-4pt} \setlength\belowdisplayskip{-6pt} % needed to make spacing more compact
\begin{multline}
\sigma(d,\alpha) = A +  B \alpha  + Cd + D\alpha^2  + E\alpha d \\ +  Fd^2  + G\alpha^3   +H \alpha^2 d   +I\alpha d^2    +Jd^3
\label{eq:model}
\end{multline}
}

%{\setlength\abovedisplayskip{-4pt} \setlength\belowdisplayskip{-6pt} % needed to make spacing more compact
\begin{table}[h]
\caption{Polynomial coefficients for the surface $\sigma(d,\alpha)$.}
\begin{center}
\begin{tabular}{|c|c|c|c|}
\hline
$A$ & $0.0125$ & $F$ & $0.0037$ \\ \hline
$B$ & $-6.0904\times 10^{-4}$ & $G$ & $3.4986 \times 10^{-8}$ \\ \hline
$C$ & $-0.0061$ & $H$ & $-3.9492 \times 10^{-6}$ \\ \hline
$D$ & $8.0999\times 10^{-6}$ & $I$ & $-2.7408 \times 10 ^{-6} $ \\ \hline
$E$ & $1.5757 \times 10^{-4}$ & $J$ & $-1.1158 \times 10^{-4}$ \\
\hline
\end{tabular}
\end{center}
\label{tab:Poly_coefs}
\end{table}
%}

\setlength\figureheight{.25\textwidth} 
\setlength\figurewidth{.4\textwidth}
\begin{figure}[h]
\centering 
\beginpgfgraphicnamed{tikzExt/distancesToPlane} 
\input{tikzFigs/distancesToPlane.tikz} 
\endpgfgraphicnamed 
\caption{ 
Histogram of the error $e_i$ for all the points $p_i$ in our data set. This shows that a normal distribution is a good approximation to our data.
} 
\label{fig:distancesToPlane}
\end{figure} 

\setlength\figureheight{0.31603\textwidth} 
\setlength\figurewidth{ 0.238  \textwidth}
\begin{figure}[p]
\centering 
\beginpgfgraphicnamed{tikzExt/numberOfPointsPerBin} 
\input{tikzFigs/numberOfPointsPerBin.tikz} 
\endpgfgraphicnamed 
\caption{ 
Number of points in each neighborhood $N_j$. In this image, each neighborhood is represented by a pixel centered at the neighborhood centroid. 
} 
\label{fig:numberOfPointsPerBin}
\end{figure} 

\setlength\figureheight{0.31603\textwidth} 
\setlength\figurewidth{ 0.238  \textwidth}
\begin{figure}[p]
\centering 
\beginpgfgraphicnamed{tikzExt/sampleMeanFigure} 
\input{tikzFigs/sampleMeanFigure.tikz} 
\endpgfgraphicnamed 
\caption{ 
Mean value of $e_i$ in each neighborhood $N_j$ which we define as $\overline{e_k}$. In our analysis we set this value equal to zero. This graph gives some experimental justification for this assumption.   
} 
\label{fig:sampleMeanFigure}
\end{figure} 

\setlength\figureheight{0.31603\textwidth} 
\setlength\figurewidth{ 0.238  \textwidth}
\begin{figure}[p]
\centering 
\beginpgfgraphicnamed{tikzExt/stdImageZeroMean} 
\input{tikzFigs/stdImageZeroMean.tikz} 
\endpgfgraphicnamed 
\caption{ 
Values for standard deviation $\sigma_j$ at each neighborhood. We can see the variance increases with larger values distance $d$ while being weakly influenced by the incident angle $\alpha$.
} 
\label{fig:stdImageZeroMean}
\end{figure} 

%\setlength\figureheight{.28\textwidth} 
%\setlength\figurewidth{.28\textwidth}
%\begin{figure}[h]
%\centering 
%%\beginpgfgraphicnamed{tikzExt/scatterWithZeroMean} 
%\input{tikzFigs/stdImageZeroMean.tikz} 
%%\endpgfgraphicnamed 
%\caption{ 
%Standard deviation of $e_i$ in each neighborhood $N_j$, using $\mu=0$
%} 
%\label{fig:stdImageZeroMean}
%\end{figure} 

% had to edit tikz file
%  - comment every other line of surface out
%  - use "hot" instead of "jet"
%  - specify "mark size=.3pt" in addplot3
\pgfplotsset{xlabel shift = -1ex}
\pgfplotsset{ylabel shift = -1ex}
\setlength\figureheight{.35\textwidth} 
\setlength\figurewidth{.32\textwidth}
\begin{figure}[p]
\centering 
\beginpgfgraphicnamed{tikzExt/scatterWithZeroMean} 
\input{tikzFigs/scatterWithZeroMean.tikz} 
\endpgfgraphicnamed 
\caption{ 
Our noise model for the Kinect. The dots represent the local estimation of $\sigma$ for a particular neighborhood $N_j$. 
} 
\label{fig:scatterWithZeroMean}
\end{figure} 


