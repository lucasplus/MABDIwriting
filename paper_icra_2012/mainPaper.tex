%%%%%%%%%%%%%%%%%%%%%%%%%%%%%%%%%%%%%%%%%%%%%%%%%%%%%%%%%%%%%%%%%%%%%%%%%%%%%%%%
%2345678901234567890123456789012345678901234567890123456789012345678901234567890
%        1         2         3         4         5         6         7         8

\documentclass[letterpaper, 10 pt, conference]{ieeeconf}  % comment this line out
                                                          % if you need a4paper
%\documentclass[a4paper, 10pt, conference]{ieeeconf}      % use this line for a4
                                                          % paper

\IEEEoverridecommandlockouts                              % this command is only
                                                          % needed if you want to
                                                          % use the \thanks command
\overrideIEEEmargins
% see the \addtolength command later in the file to balance the column lengths
% on the last page of the document

% the following packages can be found on http:\\www.ctan.org
\usepackage{graphics} % for pdf, bitmapped graphics files
\usepackage{epsfig} % for postscript graphics files
\usepackage{mathptmx} % assumes new font selection scheme installed
\usepackage{times} % assumes new font selection scheme installed
%\usepackage[fleqn]{amsmath} % assumes amsmath package installed
\interdisplaylinepenalty=500
\usepackage{amsmath}
\usepackage{amsfonts} % to get the font for real numbers

%\setlength{\mathindent}{4pt} % use to adjust indent of equations

% added in order to create pretty pseudocode - lucas
\usepackage{eqparbox}
\usepackage{algorithm,caption}
\usepackage[noend]{algorithmic}
%\renewcommand{\algorithmiccomment}[1]{\hspace{2em}// #1}
\renewcommand\algorithmicthen{} % remove then
\renewcommand\algorithmicdo{} % remove do
\renewcommand{\algorithmiccomment}[1]{ \hspace{2em} \% #1}
%\floatname{algorithm}{ }

\usepackage{textcomp}

\usepackage{microtype} % super package to make fit - lucas

\usepackage{graphicx} % to make equations larger - lucas

\usepackage{subfigure} % to place pictures side by side - lucas
%\usepackage[pdftex,hidelinks]{hyperref} % to link our dataset - csantos

\usepackage{tikz,pgfplots} % to make sweet tikz plots - lucas
\pgfplotsset{compat=newest}
\pgfplotsset{plot coordinates/math parser=false}
\newlength\figureheight
\newlength\figurewidth

% items needed to externalize graphics
\pgfrealjobname{mainPaper}
\long\def\beginpgfgraphicnamed#1#2\endpgfgraphicnamed{\includegraphics{#1}}
% pdflatex --jobname=<fig_name> mainPaper.tex

\newcommand{\vectornorm}[1]{\left|\left|#1\right|\right|} % for doing a norm of a vector in math mode - lucas

% \usepackage{cite} % it was supposed to combine multiple cites

% use to control the spacing between text and equations - lucas
%\abovedisplayskip=-5pt
%\belowdisplayskip=0pt
%\abovedisplayshortskip=-3pt
%\belowdisplayshortskip=7pt

\title{\LARGE \bf
Development of a Measurement Noise Model for Kinect-style Sensors%
% this wording was used in a well-known paper by fox
}

\author{Lucas Chavez$^{1}$, Cl\'audio dos S. Fernandes$^{2}$, James Milligan$^{3}$, \\
	Mario F. M. Campos$^{2}$, Luiz Chaimowicz$^{2}$, Ron Lumia$^{1}$, Rafael Fierro$^{1}$, and  M. Ani Hsieh$^{3}$%
\thanks{$^{1}$ Mechanical Engineering Department and Electrical \& Computer Engineering Department, University of New Mexico,
	Albuquerque, NM, USA
	{\tt\small \{lucasc,lumia\}@unm.edu},
	{\tt\small rfierro@ece.unm.edu}}
\thanks{$^{2}$ Departamento de Ci\^encia da Computa\c{c}\~ao,
	Universidade Federal de Minas Gerais,
	Belo Horizonte, MG, Brazil
	{\tt\small \{csantos,mario,chaimo\}@dcc.ufmg.br}}
\thanks{$^{3}$ Mechanical Engineering Department, Drexel University,
	Philadelphia, PA, USA ~
	{\tt\small \{milligan.james,mhsieh1\}@drexel.edu}}
%\thanks{ This work is partially supported by NSF grant OISE #1131305 and Sandia National Laboratories under PO 1179196 }
}

% CITATION FORMATING (MULTIPLE CITATIONS)
\usepackage{cite}
% NUMBER FORMATING
\usepackage[autolanguage]{numprint}
\let\n=\numprint
% MATH PACKAGE
\usepackage{mathtools}
% TODO NOTE FUNCTION:
\usepackage{color}
%\usepackage[usenames,dvipsnames,svgnames,table]{xcolor}
\usepackage{xcolor}
\definecolor{mgreen}{RGB}{34,139,34}
\newcommand\todo[1]{{\color{red} \textbf{\large \MakeUppercase{#1}}}}
% ACRONYM HANDLING
\usepackage{acronym}
	\acrodef{LIDAR}[LIDAR]{Light Detection and Ranging}
	\acrodef{TOF}[ToF]{Time of Flight}
	\acrodef{FOV}[FOV]{field of view}
	\acrodef{SLAM}[SLAM]{Simultaneous Localization and Mapping}

\begin{document}

\maketitle
\thispagestyle{empty}
\pagestyle{empty}



\begin{abstract}

% Kinect-style RGB-D sensors have provided an economical depth sensing
% technology. Since measurements from these sensors are noisy, most algorithms
% which use this technology require an underlying model to describe the
% uncertainty in the measurements. Models, up until now, have been ad hoc. This
% work will develop a measurement noise model based on thorough statistical
% analysis of a large data set. In this paper, we describe our entire methodology
% and give the generated model. Also, we validate our model with two different
% sensors. Finally, our implementation is freely available online.

% What did we do? Why? To whom is it going to be useful?

% We present an approximate measurement noise model for the Kinect sensor by using frames
% captured from a large flat wall. This work aims to provide the reader with a
% more detailed noise description than the models currently available on the
% literature. Such models are of crucial importance for probabilistic algorithms
% in the robotics field, such as stochastic environmental mapping and Monte Carlo
% Localization frameworks.

% This work looks to develop an error model for the Microsoft Kinect sensor for
% use in further research in the field of computer vision. The model takes into
% account both the distance and orientation the object within the field of
% measurement. Using data collected experimentally a 3D-surface was generated
% using the distance, orientation, and a sensor reliability weight. This sensor
% reliability weight can then be used as a means to minimize the effects of
% outliers and error in generating point cloud data using the Microsoft Kinect.

% The availability of economical, depth-based sensing technology, such as the
% Microsoft Kinect, has been a major factor in the recent advance of robotics.
% However, measurements made by these inexpensive sensors are often quite noisy .
% Current algorithms hoping to make use of this data require underlying models to
% describe this uncertainty.  This work looks to improve on the ad hoc models
% typically used by developing a statistical model based on the analysis of a
% significantly large data set. In this paper, the process used in gathering a
% representative data set for the Kinect and the statistical methodology used in
% generating a sensor model is described.  A representative function of sensor
% noise based on inputs of distance and angle of incident is presented and further
% validated with two additional data sets generated by different sensors.  The
% full implementation of this work is freely available online.

% James abstract 2
Kinect-style RGB-D sensors have provided an economical depth sensing technology.
However, measurements made with these sensors contain significant noise.  This
work aims to develop a measurement noise model for the depth values acquired
from Kinect-style sensors using a statistical analysis of a large real-world
data set.  We present the methods used for collecting a data set as well as a
summary of the statistical analysis performed to obtain a sensor model using a
Kinect sensor. Our generated model shows that uncertainty is strongly influenced
by distance and weakly influenced by incident angle. A verification of our model
was performed using a second Kinect and an ASUS Xtion PRO. The results show our
model would be a good approximation for other Kinect-style RGB-D sensors, but
that a model may need to be generated for a specific sensor to obtain higher
accuracy. Additionally, our implementation is freely available online.

\end{abstract}


\section{Introduction} \label{sec:introduction}

Many robotic applications, especially those that involve human robot interaction, often require a rich representation of the environment in order to perform such behavior as path planning and obstacle avoidance. In general, a rich representation, or map, is useful for providing situational awareness to an autonomous agent. A map is also important for applications such as teleoperation \cite{Kadous2006}.

The methodology to build this
representation is a continuously evolving subject in the field of robotics.
The origins of the research into this problem dates back roughly 25 years \cite{Lorensen1987}.
Since then the methods and the representations themselves have continued to
evolve at an impressive rate. The main catalyst behind this growth is the
advancement of sensing technologies over the same time period. In general,
sensors have continued to generate measurements at higher rates, higher
resolution, and lower cost over the years. This has provided an amazing
opportunity to build richer and more useful representations of the
environment.

In robotics map building in an unknown environment is referred to as
the Simultaneous Localization and Mapping (SLAM) problem \cite{Thrun2002}. This label
describes the fact that a methodology which solves the SLAM problem must
simultaneously locate the robot in the environment as well as map the
environment.

The focus of this work is the mapping aspect of the SLAM problem.
Early mapping methods represented
the environment as a set of landmark locations. The result was a sparse set
of points usually on a 2D plane. This allowed research to show that their
SLAM solutions worked but it soon became clear that a richer representation
of the world was needed for a growing number of applications. In response
several methods were developed using various other representations.  A
number of representations are compared in Table \ref{tab:rep}.

\begin{table}[h]
\begin{footnotesize}
\begin{center}
\begin{tabular}{|l|c|c|c|c|c|}
\hline
\multirow{2}{*}{} & Adaptability & Computationally & Low Memory & SA: & SA: \\
 & & Inexpensive & Requirement & Robot & Human \\\hline
Landmark Locations  	& x & x & x & - & - \\
Point Clouds		& - & x & - & - & - \\
Surfels             	& x & x & x & - & x \\
Implicit Functions 	& x & - & - & x & x \\
Static Mesh	 	& - & x & x & x & x \\
Adaptive Mesh	 	& x & o & o & x & x \\
\hline
\end{tabular}
\end{center}
\end{footnotesize}
\caption{Characteristics of current forms of representation}
\label{tab:rep}
\end{table}

Table \ref{tab:rep} compares the characteristics of map the
representations. Adaptability describes the ability of the representation
to correct itself given new information. Computational expense describes
how difficult it is to create and maintain a representation.  Memory
requirement describes how much memory a method must use to run.  Situation
Awareness (SA) describes how well suited a representation is for both robot
and human decision making. Robot decision making requires a representation
that can be used for such problems as obstacle avoidance. Human decision
making requires a method that can be allow an operator to intuitively
understand the state of the robot given the map. The Table is supposed to
reflect what a representation is capable of and not necessarily where the
state-of-the-art is.

A mesh based representation is arguably an extremely good choice in
comparison to the other representations. It has been used extensively by
the gaming community because it is the best for representing large
environments with the minimum memory. Also, this sort of representation
works well to increase the SA of a robot because methods for performing
physical simulations such as obstacle collision detection already exist. In
addition, a mesh based environment is a very natural method to display
information to a human operator.

Currently, the problem with mesh-based environmental mapping techniques is
that they are greedy in the sense that the mesh elements can not be
corrected using new information. Once the mesh is in place there is no
mechanism to adapt to newer measurements. The problem of adapting a mesh to
new information is a very well studied problem in computer graphics, but
these methods were not designed with large scale environmental mapping in
mind. The biggest questions are:

\begin{itemize}
\item How can we quickly decide which measurements should be used to adapt
which part of the mesh?
\item How can we quickly detect new and removed objects?
\item How can we robustly deal with noise and obtain a methodology that makes use of the new
measurements of an already existing part of the representation?
\end{itemize}

So the real question is can we develop a methodology that can address all
of the above questions and still have a manageable memory requirement and
be computationally feasible? The goal of this work is to show that MABDI is
capable of addressing these questions.

\begin{figure}[thpb]
\centering
\includegraphics[width=.5\textwidth]{figures/diagram_goal.png}
\caption{In this work, the goal of mapping is to generate  }
\label{fig:goal}
\end{figure}


\section{Related Works}	\label{sec:related_works}

Works related to MABDI are generally based on RGB-D sensors. This type of sensor has
become very popular since the release of the Kinect from Microsoft, which
was the first mass produced RGB-D sensor of its kind. RGB-D sensors are
inexpensive and produce noisy 640x480 depth images at 30fps. The RGB-D
sensor has excited the robotics community because this has been the first
time that depth data has been so readily accessible from such an
inexpensive sensor. Therefore, methodologies that use RGB-D data must be able to quickly
deal with very high rates of information.

One very impressive work came from Henry et al. in 2012 \cite{Henry2012}. In
this work they designed a system which used a RGB-D sensor to build a map made
of surfels (Surfels are circular disks which have a particular position and
orientation and also a radial size based on confidence.). In order to generate
and maintain the surfel map they used the work of Weise et al. \cite{Weise2009}.
The map consists of a large number of surfels. The surfel map can be updated
given new registered depth images from the sensor. Decisions are made how
to handle each measurement in the depth image based on the difference between an
expectation generated using the current map and the actual readings from the
sensor. Rendering a surfel map requires special methods \cite{Pfister2000} and
is difficult to use in applications such as obstacle avoidance.

One of the next major advances was published by Whelan
et al. in 2012 \cite{Whelan2012} and more recently in 2013
\cite{Whelan12tr}. The system they developed was named Kintinuous and was
able to produce a high quality mesh representation of the environment.
Their hybrid system utilized the KinectFusion method
\cite{Newcombe2011a} of Newcombe et al. to create a volumetric
representation of the portion of the environment in front of the sensor. As
the sensor moves, portions of the environment that leave the volume in
front of the sensor are ray cast and turned into a mesh. They obtain very
impressive results but also mention a limitation of their system for future
work. The limitation is that the mesh can not be updated once created,
which is an issue when revisiting parts of the environment. One of the most
impressive current works which has an adaptable mesh came from Cashier et
al. in 2012 \cite{Cahier2012}. In this work, they were able to generate and
update a mesh with new measurements from a ToF sensor. They used the
difference between the existing model and the actual measurements to decide
whether to adapt the mesh or add new elements. The mesh topology was not
adaptive to the environment and their experiments only showed results of mapping a
single flat wall with no robot movement. The system needs to be tested for
object addition and removal.

Research and development of new mapping algorithms trend towards
leveraging the information in the global map to make decisions about the
incoming data. One can see parallels with how we as humans see the world. MABDI
proposes do this in a computationally feasible way by simply using
differencing and thresholding imaging methods.


\documentclass[12pt]{article}

\usepackage{graphics}
\usepackage{epsfig}
\usepackage{times}
\usepackage{amsmath,amsfonts,amssymb}

\usepackage{subfig,float} % for sub figures

\usepackage{color, colortbl}

\definecolor{LightGray}{gray}{.9} % for table rows

\usepackage{multirow} % for the table 

\usepackage[nottoc]{tocbibind}

\usepackage{pgfgantt}

%\topmargin      0.0in
%\headheight     0.0in
%\headsep        0.0in
\oddsidemargin  0.0in
\evensidemargin 0.0in
%\textheight     9.0in
\textwidth      6.5in

\title{{\small University of New Mexico} \\ ~\\ 
Adaptive Mesh Based Surface Reconstruction For Noisy and Incremental Point Cloud Data Sets}

\author{ 
\small A thesis proposal submitted in partial fulfilment for the degree of Master’s of Science \\ ~\\
{Lucas Chavez}  \\
{\small lucasc@unm.edu} \\ ~\\
Department of Mechanical Engineering \\ 
University of New Mexico \\ ~\\
Advior \\ Dr. Ron Lumia \\ ~\\
December 2012
}

\date{}

\begin{document}
\pagestyle{plain}
\pagenumbering{roman}
\maketitle

\pagenumbering{arabic}

\section{Approach}
\label{ch:approach}

A clear idea of the algorithmic structure of the proposed system is
given by the System Flow Diagram in Figure \ref{fig:SD}. A basic
description of the main variables can be found in Table \ref{tab:var}.
The inputs to the system are RGB-D data from a Kinect-style sensor $D$
and the pose of the sensor $P$. The end goal of the system is to update
the current mesh representation in each iteration. This update is
represented by the last step in the System Flow Diagram. We can see that
the update step is primed by two distinct processes. In the diagram,
each processes is signified  by a blue background. On the left hand side
we have a processes which is named the Triangulation Process. This
process defines a triangulation $T$ based on the current depth image. On
the right hand side we have a processes which is named the
Categorization Process. This process categorizes the measurements based
on the effect they will have on the model. This triangulation
and categorization will be used by the update procedure to efficiently
evolve the map based on the current sensor measurements.                                                                 

\begin{figure}[h]
  \centering
    \includegraphics[height=0.8\textwidth]{SD.pdf}
  \caption{System Flow Diagram}
  \label{fig:SD}
\end{figure}

\begin{table}[h]
\begin{center}
\begin{tabular}{|c|l|}
\hline
{\bf Variable Name} & \multicolumn{1}{|c|}{{\bf Description}} \\
\hline
\rowcolor{LightGray} $D$ & Depth image from RGB-D sensor \\ 
$F$ & Frequency response image \\
\rowcolor{LightGray} $K_S,K_L,\text{ and }K_G$ & Image convolution operators \\
$V \text{ and } C$ & Mesh vertices and connectivity of the current triangulation \\
\rowcolor{LightGray} $T$ & Current triangulation. Contains both $V$ and $C$ \\
$P$ & The known pose of the sensor \\
\rowcolor{LightGray} $E$ & Expected depth image \\ 
$D_u,D_s,D_n,\text{ and } D_r$ & Regions of the $D$ classified by the
effect on the model \\ 
\hline
\end{tabular}
\end{center}
\caption{Basic description of the main variables}
\label{tab:var}
\end{table}

In the remaining sections of the Approach we will discuss the two
processes which prime the update step in detail. Finally, we will
discuss how the outputs are used to update the existing mesh map.

\subsection{Triangulation Process}

The goal of this process is to use the current depth image $D$ to estimate a
mesh of the current view of the of the environment. We can see a
simplified system flow chart of this process in Figure \ref{fig:SD_CT}.
Also, we can see an example of the outputs of the process using an
example input in Figure \ref{fig:O_CT}. It is important to
remember that not all of these new elements will be used in the update.
The decision of which elements from $T$ to use will be based on the
output of the Classification Process. 

\begin{figure}[h!]
  \centering
    \includegraphics[height=0.5\textwidth]{SD_CT.pdf}
  \caption{Flow Diagram of Current Triangulation Process}
  \label{fig:SD_CT}
\end{figure}

\begin{figure}[h]
\centering
\subfloat[Color Image]{\includegraphics[width=.3\textwidth]{m_photo.pdf}} \quad
\subfloat[Depth Image $D$]{\includegraphics[width=.3\textwidth]{m_depth.pdf}} \quad
\subfloat[High Frequency Response $F$]{\includegraphics[width=.3\textwidth]{m_freqn.pdf}} \\
\subfloat[After Histogramming $F$]{\includegraphics[width=.3\textwidth]{m_ihist.pdf}} \quad
\subfloat[Sampled Vertices $V$]{\includegraphics[width=.3\textwidth]{m_csamples.pdf}} \quad
\subfloat[Triangulation $T$]{\includegraphics[width=.3\textwidth]{m_ctr.pdf}} \\
\caption{An example of the Triangulation Process. The idea is to start
with a input depth image $D$ and output a triangulation $T$ which has a
topology which is adaptive to the frequency content of the scene.}
\label{fig:O_CT}
\end{figure}

\subsubsection{Frequency Response Image}

The objective of this step is to use image processing techniques to
quickly give an estimate of the frequency content in the depth image
$D$. The end product will be an image $F$ which is the same size as $D$
and will have high values in regions with high change. In order to accomplish
this we will use a sequence of image convolutions. Equation \ref{eqn:HF}
gives the mechanics of the calculation. The Sobel operator $K_S$ is
used to give a response at corners since it is a first order
differential operator. The Laplace operator $K_L$ is used to generate a
response in areas of curvature since it is a second order differential
operator. The Gaussian operator $K_G$ is used to spread the response to the
neighboring areas.   

\begin{align}
F&=(\ \mid D \ast K_S \mid + \mid D \ast K_L \mid \ ) \ast K_G \label{eqn:HF} \\
 & K_S\text{ -- small Sobel operator}  \notag  \\
 & K_L\text{ -- small Laplace operator} \notag \\ 
 & K_G\text{ -- large Gaussian operator} \notag
\end{align}

\subsubsection{Histogram and Sample for Vertices}

The objective of this section is to use the frequency response image $F$
to create a set of vertices $V$. These vertices will be defined as
locations in $D$. The idea is to have a denser number of vertices in
regions of $D$ which have a high frequency content. In order to
accomplish this, we first define regions of similar frequency content by histogramming $F$. An
example of this histogramming can be seen in Figure \ref{fig:O_CT}d.
Next we will probabilistically sample $F$ to define a
set of vertices $V$. The probability of each pixel being sampled is 
given by Equation \ref{eqn:vpr}. The probability is calculated by the
product of two different weights: $W_F$ is based on the region of $F$
where the measurement comes from and $W_A$ is the proportional area of
that region. The result of sampling for vertices can be seen in Figure
\ref{fig:O_CT}e. In addition, because the probability of picking each
pixel as a vertex can be calculated independently the process is
parallelizable.   

\begin{align}
p(u,v) = W_F(u,v)*W_A(u,v)
\label{eqn:vpr}
\end{align}

\subsubsection{Determine Connectivity}

Here we will use 2D Delaunay triangulation to define a connectivity
between the set of vertices $V$ found in the previous step. We are able
to define the connectivity in $\scriptstyle{\mathbb{R}}^2$ space because
the topology is conserved as we project the elements into
$\scriptstyle{\mathbb{R}}^3$.


\subsection{Classification Process}

The goal of the classification process is to use the difference between
the actual depth image and the expected to classify regions of the depth
image by the effect they will have on the model. In order to generate an
expected depth image $E$ we will use the existing mesh map $M$ and the
known pose of the sensor $P$ to create an artificial depth map of the of
what we expect the sensor to see $E$. We can then use image differencing
and binary blob detection to segment regions of the depth map $D$.  

\begin{figure}[h]
  \centering
    \includegraphics[height=0.5\textwidth]{SD_CM.pdf}
  \caption{Flow Diagram of Categorize Measurements Process}
  \label{fig:SD_CM}
\end{figure}

\subsubsection{Generate Expected Depth Image $E$}

We will use an existing graphics code library named OpenGL to generate
an expected depth image $E$.  A similar approach was used by Fallon et
al. in  \cite{Fallon2012}.  Figure \ref{fig:proj} is a figure from
Fallon's paper which gives a clear idea of the proposed method. The
procedure will be to give the existing mesh model $M$ and the current
pose $P$ to OpenGL and render a depth buffer. It is possible to define the
intrinsic parameters of the sensor to match the actual sensor. Figure
\ref{fig:proj}a and \ref{fig:proj}b give an example of this process.
Figure \ref{fig:proj}b represents the output of this step being the
expected depth image $E$.   

\begin{figure}[h]
  \centering
    \includegraphics[height=0.7\textwidth]{m_proj.png}
  \caption{Work from \cite{Fallon2012} which generates an expected depth image $E$
in the same way as the proposed method.}
  \label{fig:proj}
\end{figure}

\subsubsection{Find Unknown Parts of the Scene}

The objective of this step is to determine regions of the depth image
$D$ which correspond to areas of the environment which have never been
seen before. This will be accomplished by finding regions of $E$ which
have no measurement. This occurs when the ray traced through the pixel
never hits a mesh element. Regions which are unseen will be designated
as $D_U$ and the rest will be designated as $D_S$. 

\subsubsection{Find New and Removed Objects}

The objective of this step is determine if measurements from the known
region of the environment $D_S$ correspond to new $D_N$ or removed $D_R$
objects in the scene. If they do not belong to $D_N$ or $D_R$ then they
support an existing surface in the mesh map. Essentially they belong to
a supporting region $D_S$ until proven otherwise. In order to prove
otherwise we will use image differencing between the expected $E$ and
the seen parts of the environment $D_S$. We will threshold the
differenced image with $-\epsilon$ and $+\epsilon$ to make two distinct
binary images. Blob analysis will then be run on this image to determine
$D_N$ and $D_R$.  

\subsection{Update Mesh}

This is the last and most important step of the proposed method. Here
we will combine the triangulation $T$ found in the Triangulation Process
and the regions $D_N$, $D_S$, $D_R$, and $D_N$ found from the
Classification Process to efficiently update the existing mesh map $M$. 

\subsubsection{Add or Remove Mesh Elements}

If the measurements in $D$ correspond to new objects $D_S$ or to unseen
parts of the environment $D_N$, new mesh elements will need to be added.
The new elements will come from the triangulation defined in $T$. In
order to successfully merge the new elements with the existing mesh $M$
the bordering vertices of the region will need to be found and stitched.
It is proposed that this merging operation can be done in $D$. 

If the measurements in $D$ correspond to removed objects $D_R$, then the
elements are removed from $M$. This is done by marking the vertices
within the $D_R$ region and deleting them and their connections.  

\subsubsection{Adapt Existing Mesh}

Regions of the depth map which support an existing surface of the mesh
model will be used to adapt the mesh in order to better approximate the
real world. The idea of this process can be seen in Figure
\ref{fig:AM}. In this figure the real world is represented by the blue
surface in Figure \ref{fig:AM}a. The red dots represent the measurements
from $D$. The mesh $M$ is represented by the green surface. In the
figure we see a single vertex being adjusted. In Figure \ref{fig:AM}b a
set of planes are defined at the surrounding vertices and through the
measurements which correspond to this particular vertex. This
neighborhood of measurements will be found by defining a neighborhood in
$D$. In Figure \ref{fig:AM}c the vertex is adjusted. This adjustment
will be done by applying the Quadratic Error Measurement (QEM). The QEM
will adjust the vertex to minimize the distance between all planes which
were found in Figure \ref{fig:AM}b.    

\begin{figure}[h]
\centering
\subfloat[Vertex with neighboring measurements]{\includegraphics[width=.3\textwidth]{AM1.pdf}} \quad
\subfloat[Fitting a plane at neighboring vertices and through
neighborhood of measurements]{\includegraphics[width=.3\textwidth]{AM2.pdf}} \quad
\subfloat[Movement of vertex based on QEM]{\includegraphics[width=.3\textwidth]{AM3.pdf}} \\
\caption{This shows the process of adapting a single vertex with new
measurements. }
\label{fig:AM}
\end{figure}

\section{Validation}
\label{ch:validation}

In order to evaluate and quantify the effectiveness of the proposed mapping
methodology, a series of experiments will be run in simulation.  The reason
for validating through simulation is that we will have control of all
aspects of the experiment. In addition, we will have a known position of
the sensor in a known environment. This will allow us quantitatively measure
our error. Figure \ref{fig:Sim} gives an idea of how the simulation will be
accomplished. We will use a 3D modeling software named blender to create the
environment and save it to a .ply file. Then, we will use OpenGl to open
the .ply file and simulate readings from an RGB-D sensor. Finally, we will
port these measurements to Matlab where the proposed mapping algorithm will
be developed and tested. 

\begin{figure}[h]
\centering
\subfloat[Define geometry in Blender]{\includegraphics[width=.3\textwidth]{Sim1.pdf}} \quad
\subfloat[Render depth map in OpenGl]{\includegraphics[width=.3\textwidth]{Sim2.pdf}} \quad
\subfloat[Export depth image to Matlab]{\includegraphics[width=.3\textwidth]{Sim3.pdf}} \\
\caption{The steps of the simulation pipeline. The pipeline will allow
us to simulate a RGB-D sensor viewing a known environment with a known
pose.}
\label{fig:Sim}
\end{figure}

With this simulation pipeline we can design a set of experiments which will
sequentially test the abilities of the proposed mapping system. The idea
will be to start with the easiest tests first in order make sure the system
can map the environment under the most ideal conditions. Then, each
subsequent experiment will aim to test a particular part of the system. By
testing in this sequential manner we will be able to isolate and
troubleshoot problematic parts of the system. Consequently, we will gain
greater insight on the behavior of the system as a whole and the final
system will be robust. The last experiment will test the robustness of the
system with real world data. The following lists the proposed experiments
and briefly describes the intention behind each experiment. For discussion
purposes we can envision an environment which is made of a table and a can
on the table. 

\begin{enumerate}
\item Static scene; static object; static sensor \\
Here we will test the system under the most ideal conditions. We want to
see that the system will categorize all measurements as $D_S$ after the
initial mesh is created. We will also test the ability of the system to
adapt the current mesh over time.  
\item Static scene; static object; dynamic sensor \\
Here the sensor will move in the environment. For example, we can have it
pan across the table by 1 meter. We want the system to recognize
measurements which are from unseen parts of the environment and categorize
them as $D_U$. Also, we want new elements to be added in unknown regions
using the triangulation $T$ from the Triangulation Process.  
\item Dynamic scene; static object; static sensor \\
This experiment will test the ability of the system to detect new and
removed objects. We will do this by spontaneously adding and removing a
second object in the scene such as another cup on top of the table. We will
be looking for the system to successfully categorize the measurements as
either $D_N$ or $D_R$. We will then test the ability of the system to
quickly remove or add the corresponding mesh elements from the current
mesh. 
\item Dynamic scene; dynamic object; static sensor \\
This experiment will also test the ability of the system to react to new or
removed object, however the object will be moved into place over time. This
will be a much more thorough testing of the Categorization Process and the
ability to quickly add and remove elements. 
\item Dynamic scene; dynamic object; dynamic sensor \\
This experiment will test the entirety of the system in simulation. We can
have the sensor circle the table while new elements are being added and
removed. 
\item Real world data \\ 
This test will show the ability of the system to work with real world data
from a RGB-D sensor. We will make use of an open source data set which is
complete with pose information. 
\end{enumerate}

\section{Tasks}
\label{ch:tasks}

There are six major phases which will need to be completed for this
research. The following sections will list the steps which must be
completed for each phase. 

\subsection{Simulation Pipeline}

\begin{itemize}
\item Model all needed environments and object in Blender.
\item Simulate a RGB-D sensor viewing the environments using OpenGL.
\item Read in the simulated sensor output to Matlab.   
\end{itemize}

\subsection{Triangulation Process}

\begin{itemize}
\item Calculate frequency response image $F$
\item Sample $F$ to obtain vertices $V$
\item Triangulate vertices to obtain $T$
\end{itemize}

\subsection{Classification Process}

\begin{itemize}
\item Generate expected $E$ from current mesh $M$
\item Image differencing and blob analysis
\end{itemize}

\subsection{Map Update}

\begin{itemize}
\item Adaptation procedure
\item Add/remove elements
\end{itemize}

\subsection{Experiments}

Run validation experiments 1-6 as discussed in the Validation section. 

\section{Gantt Chart}
\label{ch:ganttchart}

In order to complete this work it will be necessary plan the completion of
the major tasks which were listed in the Tasks section. Figure \ref{fig:GC}
shows a Gantt Chart with the deadlines for each of the 5 major tasks and
also shows the time needed for writing the Thesis. The Simulation Pipeline
will be created first in order to have a database which will be used in
code development of the other processes. It is important to note that some
prior code development has been started and is represented by the gray
portion of the task bars in Figure \ref{fig:GC}. I plan to defend my thesis
April 15th.    

\begin{figure}[h]
\begin{center}
\begin{ganttchart}[y unit title=0.4cm,
y unit chart=0.5cm,
vgrid,hgrid, 
title label anchor/.style={below=-1.6ex},
title left shift=.05,
title right shift=-.05,
title height=1,
bar/.style={fill=gray!50},
incomplete/.style={fill=white},
progress label text={},
bar height=0.7,
group right shift=0,
group top shift=.6,
group height=.3,
group peaks={}{}{.2}]{28}
%labels
\gantttitle{February}{4}
\gantttitle{March}{16}
\gantttitle{April}{8} \\
\gantttitle{Week 1}{4} 
\gantttitle{Week 2}{4} 
\gantttitle{Week 3}{4} 
\gantttitle{Week 4}{4} 
\gantttitle{Week 5}{4} 
\gantttitle{Week 6}{4} 
\gantttitle{Week 7}{4} \\ 
%tasks
\ganttbar[progress=40]{Simulation Pipeline}{1}{4} \\
\ganttbar[progress=75]{Triangulation Process}{4}{7} \\
\ganttbar[progress=0]{Classification Process}{7}{12} \\
\ganttbar[progress=0]{Map Update}{12}{16} \\
\ganttbar[progress=10]{Validation}{16}{20} \\
\ganttbar[progress=0]{Write Thesis}{20}{28} \\
\end{ganttchart}
\end{center}
\caption{Gantt Chart}
\label{fig:GC}
\end{figure}

\end{document}




\section{Results} \label{sec:results} 

With the captured data, it was possible to fill 72.9\% of the original
neighborhoods with, at least, a threshold of 500 points. The distribution of
number of points across the $X$ space is illustrated by Figure
\ref{fig:numberOfPointsPerBin}.

We performed a quantile-quantile regression with the data shown in the
histogram from Figure \ref{fig:distancesToPlane} with a gaussian distribution,
and obtained a coefficient of determination of 0.805, which validates the
hypothesis that the error $e_i$ can be approximated by a Gaussian random
variable with zero mean.  Figure \ref{fig:sampleMeanFigure} shows the average
of the error calculated from the samples at each neighborhood $N_j$. Since most
of its values are dispersed close to zero, we can say that the approximation of
a zero-centered distribution for each neighborhood does not compromise our
model, whilst making it simpler for real applications.

Next we will show the results of our model generation. In Figure \ref{fig:stdImageZeroMean} we can see
the variance $\sigma_j$ in each neighborhood $N_j$ and observe that $\sigma_j$
increases with distance, which is a trend we expected. One important
observation is that, according to our model, the angle $\alpha$ does not play a
major role in determining the variation $\sigma_j$.  In Figure \ref{fig:scatterWithZeroMean} we can see a
scatter plot showing our generated polynomial surface $\sigma(\alpha,d)$;
$\sigma_j$ obtained from our data set; and $\sigma_j$ obtained from our
validation set. The surface represents the best fit polynomial surface through
the original estimations. The green dots represent the original estimations;
the red ones were acquired from the XtionPRO, while the black points were
computed from a different Kinect sensor.

We asses the ability of our model to describe our data by the coefficient of
determination. This allows us to determine the percentage of points whose
variation are described by $\sigma(\alpha,d)$. In Table \ref{tab:CD_coefs} we
can see the value of this coefficient for our large data set and also our
validation set. Our results show that our model could be used for other
Kinect device with a reasonable confidence, but that a new model would be needed
for the Xtion, should the application require a high accuracy model. 

%{\setlength\abovedisplayskip{-4pt} \setlength\belowdisplayskip{-6pt} % needed to make spacing more compact
\begin{table}[h!]
\caption{Coefficient of determination.}
\begin{center}
\begin{tabular}{|c|c|c|}
\hline
Original Kinect & Another Kinect & ASUS XtionPRO \\ \hline
$90.0\%\ $ & $74.7\%\ $ & $50.7\%\ $ \\ \hline
\end{tabular}
\end{center}
\label{tab:CD_coefs}
\end{table}
%}

The resulting measurement noise model from our analysis is given by (\ref{eq:model}) and the list of coefficients are in Table \ref{tab:Poly_coefs}.

{\setlength\abovedisplayskip{-4pt} \setlength\belowdisplayskip{-6pt} % needed to make spacing more compact
\begin{multline}
\sigma(d,\alpha) = A +  B \alpha  + Cd + D\alpha^2  + E\alpha d \\ +  Fd^2  + G\alpha^3   +H \alpha^2 d   +I\alpha d^2    +Jd^3
\label{eq:model}
\end{multline}
}

%{\setlength\abovedisplayskip{-4pt} \setlength\belowdisplayskip{-6pt} % needed to make spacing more compact
\begin{table}[h]
\caption{Polynomial coefficients for the surface $\sigma(d,\alpha)$.}
\begin{center}
\begin{tabular}{|c|c|c|c|}
\hline
$A$ & $0.0125$ & $F$ & $0.0037$ \\ \hline
$B$ & $-6.0904\times 10^{-4}$ & $G$ & $3.4986 \times 10^{-8}$ \\ \hline
$C$ & $-0.0061$ & $H$ & $-3.9492 \times 10^{-6}$ \\ \hline
$D$ & $8.0999\times 10^{-6}$ & $I$ & $-2.7408 \times 10 ^{-6} $ \\ \hline
$E$ & $1.5757 \times 10^{-4}$ & $J$ & $-1.1158 \times 10^{-4}$ \\
\hline
\end{tabular}
\end{center}
\label{tab:Poly_coefs}
\end{table}
%}

\setlength\figureheight{.25\textwidth} 
\setlength\figurewidth{.4\textwidth}
\begin{figure}[h]
\centering 
\beginpgfgraphicnamed{tikzExt/distancesToPlane} 
\input{tikzFigs/distancesToPlane.tikz} 
\endpgfgraphicnamed 
\caption{ 
Histogram of the error $e_i$ for all the points $p_i$ in our data set. This shows that a normal distribution is a good approximation to our data.
} 
\label{fig:distancesToPlane}
\end{figure} 

\setlength\figureheight{0.31603\textwidth} 
\setlength\figurewidth{ 0.238  \textwidth}
\begin{figure}[p]
\centering 
\beginpgfgraphicnamed{tikzExt/numberOfPointsPerBin} 
\input{tikzFigs/numberOfPointsPerBin.tikz} 
\endpgfgraphicnamed 
\caption{ 
Number of points in each neighborhood $N_j$. In this image, each neighborhood is represented by a pixel centered at the neighborhood centroid. 
} 
\label{fig:numberOfPointsPerBin}
\end{figure} 

\setlength\figureheight{0.31603\textwidth} 
\setlength\figurewidth{ 0.238  \textwidth}
\begin{figure}[p]
\centering 
\beginpgfgraphicnamed{tikzExt/sampleMeanFigure} 
\input{tikzFigs/sampleMeanFigure.tikz} 
\endpgfgraphicnamed 
\caption{ 
Mean value of $e_i$ in each neighborhood $N_j$ which we define as $\overline{e_k}$. In our analysis we set this value equal to zero. This graph gives some experimental justification for this assumption.   
} 
\label{fig:sampleMeanFigure}
\end{figure} 

\setlength\figureheight{0.31603\textwidth} 
\setlength\figurewidth{ 0.238  \textwidth}
\begin{figure}[p]
\centering 
\beginpgfgraphicnamed{tikzExt/stdImageZeroMean} 
\input{tikzFigs/stdImageZeroMean.tikz} 
\endpgfgraphicnamed 
\caption{ 
Values for standard deviation $\sigma_j$ at each neighborhood. We can see the variance increases with larger values distance $d$ while being weakly influenced by the incident angle $\alpha$.
} 
\label{fig:stdImageZeroMean}
\end{figure} 

%\setlength\figureheight{.28\textwidth} 
%\setlength\figurewidth{.28\textwidth}
%\begin{figure}[h]
%\centering 
%%\beginpgfgraphicnamed{tikzExt/scatterWithZeroMean} 
%\input{tikzFigs/stdImageZeroMean.tikz} 
%%\endpgfgraphicnamed 
%\caption{ 
%Standard deviation of $e_i$ in each neighborhood $N_j$, using $\mu=0$
%} 
%\label{fig:stdImageZeroMean}
%\end{figure} 

% had to edit tikz file
%  - comment every other line of surface out
%  - use "hot" instead of "jet"
%  - specify "mark size=.3pt" in addplot3
\pgfplotsset{xlabel shift = -1ex}
\pgfplotsset{ylabel shift = -1ex}
\setlength\figureheight{.35\textwidth} 
\setlength\figurewidth{.32\textwidth}
\begin{figure}[p]
\centering 
\beginpgfgraphicnamed{tikzExt/scatterWithZeroMean} 
\input{tikzFigs/scatterWithZeroMean.tikz} 
\endpgfgraphicnamed 
\caption{ 
Our noise model for the Kinect. The dots represent the local estimation of $\sigma$ for a particular neighborhood $N_j$. 
} 
\label{fig:scatterWithZeroMean}
\end{figure} 




\section{Conclusion} \label{sec:conclusion} 

% this work presents 
% by ...

% needs:
% created a measurement noise model 
% cheap method to estimate "ground truth"
% generated model which would will be more accurate then ad hoc models
% our model should work for other Kinect sensors reasonably well
% maybe need to regenerate if have an ASUS
% we provide our methodology so that one can do that
% future work - apply to KinectFusion

This work has presented a novel methodology for generating a measurement noise
model for Kinect-style sensors.  Classical work in sensor modeling has defined
the uncertainty to be constant for all measurements. For many depth sensors,
especially those like Kinect, this constant assumption simply isn't true. In
recent work ad hoc models have been created to be more realistic. Our method
generates sensor models from a thorough stochastic analysis of a large data
set. We give our generated model and compare the results with a validation set.
Our method is better than an ad hoc guess, however a model may need to be
generated for a specific sensor depending on the needs of the algorithm it will
be used for. We have made it easy to generate a model for your own sensor by
making the code freely available. In future work we intend to improve our model
by taking into account the error induced by lens distortion. Additionally, we
would like to evaluate our model with existing SLAM algorithms.


\section*{Acknowledgment}
The authors would like to thank professors Ant\^onio W. Vieira and Renato
Assun\c{c}\~ao for their valuable theoretical input. This work is partially
supported by NSF grant OISE \#1131305 and Sandia National Laboratories under PO
1179196, and by the brazilian institutes CAPES, CNPq and FAPEMIG.

\bibliographystyle{IEEEtran}
\bibliography{bibliography}

\end{document}
