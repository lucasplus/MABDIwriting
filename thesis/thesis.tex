
\documentclass[botnum, nobox]{unmeethesis}

\usepackage{float}                 % to force figure placement with "H"

\usepackage{amsmath}               % to show matrices \begin{bmatrix}

\usepackage{blindtext}             % Sometimes I get the black boxes seen here http://tex.stackexchange.com/a/111956
\usepackage{multirow}              % for the table
\usepackage{color, colortbl}
\definecolor{LightGray}{gray}{.9}  % for table rows
\usepackage{relsize}
\usepackage{graphicx}
\usepackage{caption}     % for multi figure figures
\usepackage{subcaption}  % for multi figure figures

\usepackage{listings}              % for appendix of python code

\definecolor{mygreen}{rgb}{0,0.6,0}
\definecolor{mygray}{rgb}{0.5,0.5,0.5}
\definecolor{mymauve}{rgb}{0.58,0,0.82}

\lstset{ %
  backgroundcolor=\color{white},   % choose the background color; you must add \usepackage{color} or \usepackage{xcolor}
  basicstyle=\scriptsize,          % the size of the fonts that are used for the code
  breakatwhitespace=false,         % sets if automatic breaks should only happen at whitespace
  breaklines=true,                 % sets automatic line breaking
  captionpos=t,                    % sets the caption-position to bottom
  commentstyle=\color{mygreen},    % comment style
  deletekeywords={...},            % if you want to delete keywords from the given language
  escapeinside={\%*}{*)},          % if you want to add LaTeX within your code
  extendedchars=true,              % lets you use non-ASCII characters; for 8-bits encodings only, does not work with UTF-8
  frame=single,	                   % adds a frame around the code
  keepspaces=true,                 % keeps spaces in text, useful for keeping indentation of code (possibly needs columns=flexible)
  keywordstyle=\color{blue},       % keyword style
  language=Python,                 % the language of the code
  otherkeywords={*,...},           % if you want to add more keywords to the set
  numbers=left,                    % where to put the line-numbers; possible values are (none, left, right)
  numbersep=5pt,                   % how far the line-numbers are from the code
  numberstyle=\tiny\color{mygray}, % the style that is used for the line-numbers
  rulecolor=\color{black},         % if not set, the frame-color may be changed on line-breaks within not-black text (e.g. comments (green here))
  showspaces=false,                % show spaces everywhere adding particular underscores; it overrides 'showstringspaces'
  showstringspaces=false,          % underline spaces within strings only
  showtabs=false,                  % show tabs within strings adding particular underscores
  stepnumber=1,                    % the step between two line-numbers. If it's 1, each line will be numbered
  stringstyle=\color{mymauve},     % string literal style
  tabsize=2,	                     % sets default tabsize to 2 spaces
  title=\lstname                   % show the filename of files included with \lstinputlisting; also try caption instead of title
}

\captionsetup[table]{skip=0pt} % reduce caption space between tables and caption

\begin{document} % _____________________________________________ Begin Document

\frontmatter

% Uncomment the next command if you see weird paragraph spacing:
% That is, if you see paragraphs float with lots of white space
% in between them:
\setlength{\parskip}{0.30cm}


\title{Mesh Addition Based on the Depth Image\\(MABDI)}

\author{Lucas E. Chavez}

\degreesubject{M.S., Mechanical Engineering}

\degree{Master of Science \\ Mechanical Engineering}

\documenttype{Thesis}

\previousdegrees{B.S., Mechanical Engineering \\
New Mexico Institute of Mining and Technology, 2009}

\date{December, 2016}

\maketitle

% \begin{dedication}
%    To my parents, Albert II and Gladys, for their support,
%    encouragement and the Corvette they're giving me for graduation. \\[3ex]
%    ``A bird in hand is worth two in the bush''
%          -- Anonymous
% \end{dedication}

\begin{acknowledgments}
   \vspace{1.1in}
   This work was supported in part by Sandia National Laboratories under Purchase
   Order: 1179196 and  NSF grant OISE \#1131305.
\end{acknowledgments}

\maketitleabstract %(required even though there's no abstract title anymore)

\begin{abstract}
  Many robotic applications, especially those whose goal is to aid or assist
  through human-robot interaction, utilize a rich map of the world for reasoning
  tasks such as collision detection, path planning, or object recognition. Such
  a map, and the method used to produce it, must take into consideration
  real-world constraints. Most mesh-based mapping algorithms resemble a ``black
  box'' and do not provide a mechanism to close the loop and make decisions
  about the incoming information. MABDI leverages the global mesh by finding the
  difference between what we expect to see and what we are actually seeing, and
  using this to classify the incoming measurements as novel or not. This allows
  the surface reconstruction method to be run only on data that has not yet been
  represented in the global mesh. The result is an algorithm that becomes
  computationally inexpensive once the environment is known, but still can
  incorporate new objects into the model.
\clearpage %(required for 1-page abstract)
\end{abstract}

\tableofcontents
\pagebreak

\mainmatter % _____________________________________________________ Main Matter


\chapter{Introduction} \label{chapter:introduction}
\section{Overview}

% Overview
% * applications need a map
% * methods have been evolving, fueled by new sensors. In particular RGB-D
%   RGB-D Sensors
%   * Specs of RGB-D sensors, create lots of data
%   * Mapping methods must be able to handle this data
%   Maps
%   * slam problem, we are concerned with mapping
%   * types of maps, we are concerned with rich types
%   * list of constraints: supported, computation, memory
%   * discussion of why mesh
% Goal
% * Create a system that can intelligently make decisions about sensor data
%   * Using tools already developed for mesh
%   * Computationally feasible
%   * Leverage information that we already know
% Contribution
% * Discussion of difference with "black box" methods

% Overview
% * applications need a map
% * methods have been evolving, fueled by new sensors
% * slam problem, we are concerned with mapping
% * types of maps, we are concerned with rich types
% * list of constraints: supported, computation, memory
% * discussion of why mesh
% Contribution
% * Discussion of difference with "black box" methods

Many robotic applications, especially those that involve human-robot
interaction, often require a rich representation of the environment in order to
perform such behavior as path planning and obstacle avoidance. In general, a
rich representation, or map, is useful for providing situational awareness to an
autonomous agent. A map is also important for applications such as teleoperation
\cite{Kadous2006}.

In robotics, map building in an unknown environment is referred to as the
Simultaneous Localization and Mapping (SLAM) problem \cite{Thrun2002}. This
label describes the fact that a methodology which solves the SLAM problem must
simultaneously locate the robot in the environment as well as map the
environment. The focus of this work is the mapping aspect of the SLAM problem.
Fig. \ref{fig:goal} gives a visualization of the goal.

\begin{figure}[h]%[thpb]
\centering
  \includegraphics[width=.75\textwidth]{figures/diagram_goal.png}
  \caption{Goal is to create a map from depth images.}
  \label{fig:goal}
\end{figure}

The methodology to build a map is a continuously evolving subject in the field
of robotics and computer graphics. Well known works of map building methods
began to be seen around 1987 \cite{Lorensen1987}. Since then, the methods and
the representations themselves have continued to evolve at an impressive rate.
Growth in this field of research has been fueled by continuous advances in
computing and sensing technologies. Over the years, sensors have continued to
generate measurements at higher rates, higher resolution, and lower cost. RGB-D
sensors are a new category of sensor that have recently gained extensive
popularity in the robotics community due to their affordability and ability to
generate a rich amount of data.

\subsection{RGB-D Sensors}

The popularity of RGB-D sensors began with the release and commercialization of
the Kinect\texttrademark ~ by Microsoft. The arrival of the Kinect brought with
it an inexpensive depth sensor that uses an active range system to generate a
depth map of a given environment \cite{freedman2012depth}. The Kinect and
similar sensors, have come to be called RGB-D sensors. This class of sensors
provide images which include both visual (RGB) and depth (D) values. Several
works have taken advantage of this sensor technology in scenarios such as
environmental mapping \cite{henry2012rgb}, 3D reconstruction
\cite{Newcombe2011a}, gesture recognition \cite{Xia2011}, and altitude control
of aerial vehicles \cite{Stowers2011}.

RGB-D sensors generally provide data at 30 frames per second and 640$\times$480
resolution. Consequently, methods that use RGB-D data must handle over 9 million
pixel values per second, if only using the depth information (D), and over 18
million if using both color (RGB) and depth (D). The magnitude of the amount of
data output from RGB-D sensors creates the need for mapping methods that are
computationally inexpensive and also influences the type of data structure used
to store the map.

\subsection{Maps}

There are different types of data structures that can define a map. All types
have both intrinsic characteristics that impact the algorithms that generate
them and constraints that must be considered for real-world applications. In
addition, we are concerned with rich representation types, in contrast to sparse
representation types \cite{Dissanayake2001}, because rich types have the most
use in applications such as human-robot interaction.

\begin{table}[h]
  \caption{Comparison of constraints for different map types.}
  \label{tab:rep}
  \begin{footnotesize}
  \begin{center}
    \begin{tabular}{|l|c|c|c|c|c|}
    \hline
    \multirow{2}{*}{}   & Supported & Computationally & Low Memory  \\
                        &           & Inexpensive     & Requirement \\\hline
    Point Clouds		    & x         & x               & -           \\
    Surfels             & -         & x               & x           \\
    Implicit Functions 	& x         & -               & -           \\
    Mesh	 	            & x         & x               & x           \\
    \hline
    \end{tabular}
  \end{center}
  \end{footnotesize}
\end{table}

When considering which type of map is best for real-world applications, we must
consider the constraints imposed by each type:

\begin{itemize}
  \item Supported - Is there software, tools, research, algorithms, etc., for
  this type of map?
  \item Computationally Inexpensive - Can the algorithms run quickly on low cost
  computers (rather than specialized hardware)?
  \item Low Memory Requirement - Can the algorithms run on hardware with
  a standard amount of RAM?
\end{itemize}

Table \ref{tab:rep} compares the constraints of common map types. We can see, in
general a mesh type map satisfies real-world constraints. Additionally, meshes
have been used extensively by the gaming and graphics communities, and so
benefits from an incredible amount of continued research and advances in
hardware such as Graphics Processing Units (GPUs).

\section{Goal}

The goal of this work is to develop a mapping algorithm that can gracefully
utilize the amount of data output from an RGB-D sensor. Additionally, the
algorithm will make use of software tools and hardware that have been developed
for mesh data structures. The algorithm will be able to make intelligent
decisions using the data it receives based on the knowledge it has been building
about the environment. The decisions will be driven by the leveraging the
difference between what the algorithm is actually seeing and what it expects to
see. The decisions will be generated using computationally inexpensive computer
vision methods.

\section{Contribution}

MABDI's contribution to the state-of-the-art in mesh based environmental mapping
methods is to close the loop of the algorithmic structure used by current
methods. Current mesh mapping techniques can be described as ``black box''
methods. The general algorithmic structure of these ``black box'' methods
follows the same pattern, which can be thought of as a pipeline. Data comes in
from the sensor, those measurements are used to create a mesh, and then that
mesh is appended to a global mesh. Fig. \ref{fig:pipeline_blackbox} shows this
pipeline. We can then compare that pipeline to the pipeline used in MABDI, which
is shown in Fig. \ref{fig:pipeline_mabdi}. In both figures we can see the
``Create Mesh from Input'' component. The input to this component is different
for ``black box'' methods and MABDI. ``Black box'' methods input all data from
the sensor into the mesh creating component whereas MABDI only inputs data
identified to be from the unknown parts of the environment. This reduces the
computational cost of the mesh creating component and results in a global mesh
that does not have redundant mesh elements. Redundant mesh elements are an
inherent problem of ``black box'' methods and are typically removed using
computationally expensive post processing of the global mesh. MABDI simply does
not create redundant elements to begin with. Reduced computational cost and a
global mesh that is inherently free of redundant elements is MABDI's
contribution.


\begin{figure}[h]%[thpb]
\centering
\begin{subfigure}[t]{\textwidth}
  \includegraphics[width=.9\textwidth]
    {\detokenize{figures/diagram_general_pipeline_blackbox.png}}
  \caption{``Black box'' methods}
  \label{fig:pipeline_blackbox}
\end{subfigure}
\begin{subfigure}[b]{\textwidth}
  \includegraphics[width=.9\textwidth]
    {\detokenize{figures/diagram_general_pipeline_mabdi.png}}
  \caption{MABDI}
  \label{fig:pipeline_mabdi}
\end{subfigure}
\caption{Algorithmic structure of ``black box'' methods (a) compared to MABDI
(b). Contribution of MABDI to the state-of-the-art shown in red}
\label{fig:pipeline}
\end{figure}


\chapter{Related Works}	\label{chapter:related_works}

% Works related to MABDI are generally based on RGB-D sensors. This type of sensor has
% become very popular since the release of the Kinect from Microsoft, which
% was the first mass produced RGB-D sensor of its kind. RGB-D sensors are
% inexpensive and produce noisy 640x480 depth images at 30fps. The RGB-D
% sensor has excited the robotics community because this has been the first
% time that depth data has been so readily accessible from such an
% inexpensive sensor. Therefore, methodologies that use RGB-D data must be able to quickly
% deal with very high rates of information.
%
% Research and development of new mapping algorithms trend towards
% leveraging the information in the global map to make decisions about the
% incoming data. One can see parallels with how we as humans see the world. MABDI
% proposes do this in a computationally feasible way by simply using
% differencing and thresholding imaging methods.


A major problem in robotics has been and continues to be: How can we create the
``best'' representation of an unknown environment? There are two main
communities of researchers who been working on developing algorithms and methods
to answer precisely this question. They are the robotics community and the
computer graphics community and each community has a slightly different
motivation for solving this problem. The robotics community is concerned with
developing a real-time solution of generating representations in large
environments. These representations are used by both fully autonomous and
teleoperated systems. The common name which is used by the robotics community
for this problem is Simultaneous Localization and Mapping or SLAM. The name SLAM
refers to the problem of mapping and locating a robot in an unknown environment.
Early methods generated very sparse representations of the world, as time and
sensor technology progressed the representations became denser. A dense
representation is desired for any system which must have good situational
awareness of its environment. The computer graphics community is concerned with
generating high quality representations of smaller environments and single
objects. They generally refer to the problem as surface reconstruction. These
representations are used by augmented reality, computer game object creation, 3D
printing, and more applications. In the following sections we will trace the
development of representation generating methods in both communities. We will
then discuss what is needed in future work.

\section{SLAM}

The problem of SLAM has been a primary focus of the robotics community for
more than 25 years. A complete solution to the SLAM problem must be able to
generate a representation of an unknown environment and track the robot in
this new representation. In this body of literature the act of generating a
representation is referred to as mapping. A good overview of the problem
can be found in \cite{Durrant-Whyte2006} and \cite{Bailey2006}. Each
solution is designed to consider the goal application, type of sensor,
computational constraints, and memory limits. All these factors influence
the researcher's choice of what type of representation to use for the
mapping procedure. In 2002 Thrun wrote a famous survey \cite{Thrun2002} of
the SLAM literature which categorized existing algorithms on many traits
including the representation. The representation choice of prior work can
be roughly categorized into 3 types. The first type is characterized by
some sort of list of 2D or 3D points and are usually considered to be
sparse representations. Common names for these types are landmark locations
and point clouds. The second type are considered to be more volumetric
based and are often times considered to be a dense representation. Common
names for these types are occupancy grid and Truncated Signed Distance
Function (TSDF). The last type have the characteristic of being a surface
representation and are also considered to be a dense representation. Common
names for these types are surfels and mesh. In the following sections we
will trace the history of each of the 3 types of representation that is
seen in the SLAM literature.

\subsection{Point Locations}

One of the most well known and earliest solution to the SLAM problem, which
uses a point location representation, was proposed by Smith et al. in 1990
\cite{Smith1990}. The mathematical framework that he created was the origin
of a family of solutions based on the Extended Kalman Filter (EKF). The
representation he chose was simply a list of 2D landmark locations. Each
location was part of a state matrix which was estimated at every iteration.
A list of landmark locations was chosen because it allowed the method to
have a low computational cost and use a small amount of memory, important
factors in the days of early computing. There have been many improvements
to the family of SLAM solutions which generate a list of point locations
since Smith's work. One of the first practical implementations on a real
robot was done by Thrun in 1998 \cite{Thrun1998}. In this work the SLAM
problem was posed in an Expectation Maximization (EM) framework which is
similar to the EKF framework in that landmark locations are saved
in a state vector which is estimated at every iteration. In Thrun's work an
occupancy grid map is generated as a post processing step from sonar
measurements. The results showed that their representation could become
more accurate over time by using new observations to improve the current
estimate. This a highly desired ability of any representation generation
method. The next step was the ability of these methods to include a loop
closure procedure. A loop closure procedure was proposed by Gutmann in 1999
\cite{Gutmann1999}. The key ability of the method was it could recognize
when the robot was revisiting a prior location and adjust the entire
representation with the constraint that the two points must coincide. In
2001 Dissanayake et al. \cite{Dissanayake2001} derived 3 theorems to
theoretically prove the convergence of the SLAM problem. Their test
platform used a millimeter-wave radar mounted on a vehicle and generated a
list of 2D landmark locations. In 2001 Thrun et al. \cite{Thrun2001} cast
the SLAM problem using particle filter techniques. Their results generated
a 2D map and showed an increased robustness and lower computational cost
than prior methods. One of the key disadvantages of methods up to this
point was that complexity scaled quadratically with the number of landmark
locations.  In 2002 Montemerlo et al. \cite{Montemerlo2002} created a SLAM
solution named FastSLAM which was able to handle a much larger number of
landmarks. They showed results with maps containing more than 50,000
points. Then, SLAM solutions using point locations became much more
directed towards 3D.

Some of the first interesting works which represented the world as a list
of 3D point locations were done by Thrun et al. in 2000 \cite{Thrun2000},
Liu and Emery in 2001 \cite{Liu2001}, and H\"{a}hnel et al. in 2003. In
these works the 2D landmark locations and robot position were estimated
using very similar techniques from past work. Once this had been done the 3D
laser scan data was simply appended to each estimated robot location. Then,
a mesh was created by post processing the 3D point cloud. They utilized the
fact that the laser collected the data in an incremental manner and simply
connected neighboring 3D points. Finally, the mesh was simplified by
looking for large planar sections and merging the corresponding mesh
elements. One of the first SLAM solutions which used a single camera to
generate a list of 3D points was done by Davison in 2003
\cite{Davison2003}. Here he used a single camera to generate a very sparse
list of 3D points. This method was limited to small environments. Future
advances allowed representations of larger environments. In 2003 Thrun et
al.  \cite{Thrun2003} created a SLAM procedure which did not rely on having
a structured environment and was applied to mapping large mines. In 2004
Howard et al. \cite{Howard2004} created a SLAM systems based on a Segway
platform equipped with a 3D laser which could map large areas of roughly 0.5
km on each side.  One of the results showed a map with approximately 8
million points. In 2006 Cole and Newman \cite{Cole2006} continued work in
large-scale SLAM by increasing robustness and also generated maps with many
3D points using a laser sensor. In 2007 Clemente et al. created a
large-scale SLAM system that used a single camera. The system had an
advanced loop closing procedure based on visual features and created large
maps of 3D points. In 2001 Klein and Murray \cite{Klein2007} developed a
SLAM solution which used a single camera. The uniqueness of their
method was the algorithmic structure. Their SLAM solution consisted of 2
separate processes: a tracking processes and a map building process. This
algorithmic structure has become very common in many current SLAM solutions
because of the advances in pose estimation technology. Klein and Murray
were able to get very good results for a small environment and showed
Augmented Reality (AR) applications.  Many of the future advances of SLAM
solutions, which generated 3D point sets, dealt with camera systems
\cite{Paz2008,Konolige2008,Strasdat2010} and improved in speed and
robustness. Many of the most current methods which produce a list of points
are systems that use a relatively new type of sensor named a RGB-D sensor.
One good example is a work that was produced in 2011 by Engelhard et al.
\cite{Engelhard2011}. In this work they used an algorithm named the
Iterative Closest Point (ICP) \cite{Rusinkiewicz} to align point clouds
from coming from the RGB-D sensor into a large colored point cloud.  The
resulting maps were visually impressive. However, the map could not be
adapted to new information and was not well suited for other applications,
such as obstacle avoidance. These limitations are inherent in maps that
consist of lists of points.

\subsection{Volumetric}

Many SLAM solutions generate a 2D volumetric representation of the world
because they are especially advantageous in dealing with noisy sensors. Two
of the first major works which generated a 2D volumetric representation
were done in 1998 by Yamauchi et al. \cite{Yamauchi1998} and Schultz et al.
\cite{Schultz1998}. These works generated a 2D occupancy grid which is a
type of volumetric representation. Here the environment was divided into a
2D grid.  Each square of the grid contained the probability that it was
occupied with an object. All squares would be updated iteratively based on
the current sensor readings. Occupancy grids, like any other
volumetric-based representation, are limited by the amount of available
memory. In 2002 Biswas et al. \cite{Biswas2002} extended occupancy grid
methods by allowing dynamic environments. This was done by looking at past
``snapshots'' of the map. In 2004 Eliazar and Parr \cite{Eliazar2004}
continued the advancement by decreasing computational cost and implemented
a loop closure method.

There have been a few impressive SLAM solutions which generate a 3D
volumetric representation. There are three major works which generated
something very similar to a 3D occupancy grid which was saved as in a
octree data structure
\cite{Magnusson2007,Nuchter2007,Huang2011,Endres2012}. Each work had a
slightly different name and procedure for generating the representation,
but in general the representations divided the environment into cubes and
had a scalar value representing the belief of a surface being there.
Octrees were used to save memory by only having a fine resolution of cubes
at places where there was a surface. There are many advantages to a 3D
occupancy grid representation. When applied to obstacle avoidance and path
planning algorithms. Also, the representation is very adaptable to new
information. The major disadvantage is that the representation can not be
visualized immediately. In order to render, an image must be generated at
each desired viewpoint by ray tracing the volume. This can be a problem
when using such method for applications such as teleoperation due to the
computational cost of rendering. The current state of the art for
generating a volumetric representation was done by Newcombe et al.  in 2011
\cite{Newcombe2011a}.  Their system used a RGB-D sensor and generated a 3D
voxelized grid Truncated Signed Distance Function (TSDF) of the
environment. For this type of representation each cube contains the value
of the distance to the nearest surface. The sign of the value is based on
which side of the surface the cube is relative to the sensor. This work has
been the most capable at dealing with extremely noisy data and dynamic
scenes. However, due to memory constraints the method can only represent
environments that are about the size of a 4m cube. Also, it must be ray
traced in order to be visualized.

\subsection{Surface}

One of the first major works which created a surface representation of the
environment in real-time was done by Martin and Thrun in 2002
\cite{Martin2002}. Their method utilized an EM framework to fit plane
models to 3D point cloud data. Polygon mesh elements were then easily
assigned to each plane. The main drive behind this work was to generate a
map of the environment that uses a low amount of memory. Their  method
worked well for structured environments. One of the major limitations of
their method, and other methods that only mesh large planar sections, is
that the representation will only consist of planar sections and not
capture the fine detail of the environment. In 2004 Viejo and Cazorla
\cite{springerlink:10.1007/978-3-540-30463-0_30} developed a methodology
for generating a mesh that can contain more information of the environment
than large planar sections. Due to this ability, they termed their method
to be ``unconstrained.'' Essentially their method was based on a 3D
Delaunay triangulation algorithm. Giesen surveyed Delaunay triangulation
methods in \cite{Giesen2004}. Viejo and Cazorla were not able to obtain
real-time results and, in fact, it has been seen that it is extremely
difficult to run a 3D Delaunay triangulation in real time because of the
numerous amount of distance calculations that are needed.  One of the next
major advances came from Weingarten and Siegwart in 2006
\cite{Weingarten2006}. Their work also created a mesh that was only capable
of capturing large planar surfaces. However, they showed increased
robustness. In 2007 Pollefeys et al. published a work with multiple
researchers \cite{Akbarzadeh2006,Pollefeys2007}. They developed a large
urban mapping system consisting of a vehicle and eight camera systems. The
processing was carried out by multiple CPUs and optimized with Graphics
Processing Unit (GPU) calculations. In their work they used the camera
systems to generate depth maps. The set of depth maps was then fused to
create a new smaller set of depth maps that were more accurate. The depth
maps in the smaller set were more accurate because the error was averaged
out by using near-by depth maps. The smaller set was then used by a
triangulation procedure to create a mesh of the environment.  The mesh
generation procedure was based on a work from 2002 by Pajarola et al.
\cite{Pajarola2002}. This method defines a mesh in the depth image. It
starts from a very coarse mesh and continues to refine in areas of the
depth image based on a confidence criteria. In the work of Weingarten and
Siegwart these meshes which are defined for each fused depth image are then
checked for overlaps and duplicates are removed to make a single large
mesh. One of the major drawbacks of this approach is that the output mesh
can not be adapted by measurements which come from revisited parts of the
scene. Another major advancement came in 2008 from Poppinga et al.
\cite{Poppinga2008}. In this work they used a Time of Flight (ToF) camera
to generate a mesh representation of the large planar structures in the
environment. Here they also develop a procedure to determine a mesh in a
depth image. They leverage the structure of the depth image to make the
method computationally inexpensive. In their work they simply append the
meshes which are created from each depth image into a global coordinate
system. They obtain very good results from a simple method.  Once
again, the method is not adaptive to new information. Also, a mesh is
created for each depth image instead of updating and maintaining a global
mesh. A major advancement came from a famous work done by Newcombe and
Davison in 2010 \cite{Newcombe2010}. In this work they designed a method to
create a mesh reconstruction from a single video camera. Their method used
Structure From Motion (SFM) to obtain a sparse point cloud of the scene.
Then an implicit function was fit to the point cloud using the methodology
of Ohtake et al.  \cite{Ohtake2003}. A bundle of depth maps is then
selected. From the bundle a single reference depth image is selected and a
``base'' model is constructed by sampling the implicit surface for vertices
in the reference frame. The neighboring frames are used to better the
``base'' model and create a more accurate mesh. Each reference frame has
its own mesh and all the meshes are put into a global coordinate system.
Duplications are then detected and removed. Again, the representation is
not adaptive to new information. In 2010 St\"{u}hmer et al.
\cite{Stuhmer2010} advanced the field by publishing a method to generate
very accurate depth maps from several color images in real-time. They
showed very impressive results but their method was not designed to maintain
a representation in a global coordinate frame.

The next major advances in methods that generated surface representations of the
environment, were based on RGB-D sensors. This type of sensor has become very
popular since the release of the Kinect from Microsoft which was the first mass
produced RGB-D sensor of its kind. RGB-D sensors are inexpensive and produce
noisy 640x480 depth images at 30Hz. The RGB-D sensor has excited the robotics
community because this has been the first time that depth data has been so
readily accessible from such an inexpensive sensor. Therefore, these
methodologies must be able to quickly deal with very high rates of information.
One impressive work came from Henry et al. in 2012 \cite{Henry2012}. In this
work they designed a system which used a RGB-D sensor to build a map made of
surfels (Surfels are circular disks which have a particular position and
orientation and also a radial size based on confidence.). In order to generate
and maintain the surfel map they used the work of Weise et al. \cite{Weise2009}.
The map consists of a large number of surfels. The surfel map can be updated
given new registered depth images from the sensor. Decisions are made how to
handle each measurement in the depth image based on the difference between an
expectation generated using the current map and the actual readings from the
sensor. Rendering a surfel map requires special methods \cite{Pfister2000} and
is difficult to use in applications such as obstacle avoidance.

One of the next major advances is a highly-related work that was published by
Whelan et al. in 2012 \cite{Whelan2012} and more recently in 2013
\cite{Whelan12tr}. The system they developed was named Kintinuous and was able
to produce a high quality mesh representation of the environment. Their hybrid
system utilized the KinectFusion method \cite{Newcombe2011a} of Newcombe et al.
to create a volumetric representation of the portion of the environment in front
of the sensor. As the sensor moves, portions of the environment that leave the
volume in front of the sensor are ray cast and turned into a mesh. They obtain
very impressive results but also mention a limitation of their system for future
work. The limitation is that the mesh can not be updated once created, which is
an issue when revisiting parts of the environment. One of the most impressive
current works which has an adaptable mesh came from Cashier et al. in 2012
\cite{Cahier2012}. In this work, they were able to generate and update a mesh
with new measurements from a ToF sensor. They used the difference between the
existing model and the actual measurements to decide whether to adapt the mesh
or add new elements. The mesh topology was not adaptive to the environment and
their experiments only showed results of mapping a single flat wall with no
robot movement. The system needs to be tested for object addition and removal.

\section{Surface Reconstruction}

The computer graphics field has spent considerable effort to develop methodologies
for creating representations from sets of data. Generally, these sets of
data are acquired from a sensor. Methodologies have progressed steadily and
are often designed for a specific application. One of the original motivations
was to generate surfaces from medical imaging data. This
allows doctors to make better decisions because the data are presented in a
more intuitive manner. Current applications include augmented reality and
3D printing. Older methodologies were not as concerned with speed and often
times had a large computational cost. Also, the methodologies are often
designed for single objects or small environments. Following the taxonomy
of such well-known works as \cite{Gopi2002,Mencl1997}, the field can be
roughly divided into representations that are generated with volume-based
techniques and those that use surface-based techniques. Methods that use
volume-based techniques are characterized by spatially subdividing the
environmental volume and are usually computationally expensive and require
a large amount of memory. Methods that use surface-based techniques
generate the representation using surface properties of the input data.
Both types of methods can have mechanisms to adapt the mesh to noisy or new
information.  In the following section we will trace the progression of the
methodologies.

\subsection{Volume-based}

Volume-based methods spatially subdivide the volume into smaller parts and
operations are performed on the mesh to either implicitly imply the surface
or define the surface depending on the information contained in each
subdivision. One of the first well-known works that used a volume-based
technique was proposed by Lorensen and Cline in 1987 \cite{Lorensen1987}.
In this work they proposed a method named marching cubes which is still
known for its reliability and simplicity and is used by applications which
do not have a computational requirement. Marching cubes subdivides the
space into cubes. The data contained in each cube dictate how the surface
connectivity will be defined in that cube. Possible vertex locations are at
the corners and along the edges. Once this has been done for all cubes the
process is complete. One of the next major steps came from Hoppe et al. in
1992 \cite{Hoppe1992} In this work they used the input points to define a
Signed Distance Function (SDF) in 3D space and then meshed the zero-set to
obtain the output mesh. A SDF is a spatial function which has the value of
the distance to the nearest surface at each point. The sign is used to
specify if the point is inside our outside of the surface relative to the
sensor. The zero-set of the SDF is the surface where the values transition
from positive to negative. Using a SDF has proven to be very effective and
has been the core idea of many methodologies that came after this work of
Hoppe et al., such as KinectFusion \cite{Newcombe2011a}. One of the next
advances came from Edelsbrunner and M\"{u}cke in 1994
\cite{Edelsbrunner1994} with a method named alpha shapes. Here they used 3D
Delaunay triangulation and the input point set to decompose the volume into
a Delaunay tetrahedrization. This gives a triangulation of the input set
which involves all points. A sphere of radius alpha is then used to remove
edges and vertices to obtain a mesh of user specified resolution. Many
works have made use of 3D Delaunay triangulation to create a mesh. Methods
which use 3D Delaunay on the input set have a large computational cost and
often cannot be executed in real-time.  The next valuable contribution came
from Bloomenthal in 1994 \cite{Bloomenthal1994} as open source software for
surface polygonization of implicit functions. This was a stable and robust
open source software that has been used in many well-known algorithms
\cite{Newcombe2010}.  Another major advance came from Curless and Levoy in
1996 \cite{Curless1996}. In this work they also constructed a Truncated
Signed Distance Function (TSDF). A TSDF is very similar to a SDF, the only
difference is that distance values are truncated after they exceed a
certain threshold. Their method was one of the first to be able to handle
several registered range scans.  Their work showed how well a TSDF can deal
with several noisy scans by naturally integrating out the error. They
obtained very good results but not even close to real-time. A speed up in
processing time was achieved by Pulli et al. in 1997 \cite{Pulli1997} by
utilizing octrees. They obtained good results and their method was used by
Surmann et al. \cite{Surmann2003} in a well-known robotic mapping work.
Another major advance came in 2001 from Zhao et al \cite{Zhao2001}. They
used Partial Differential Equation (PDE) methods to obtain a final
reconstruction that was of better quality than prior methods. In 2001 Carr
et al. \cite{Carr2001} created a volumetric method based on the radial
basis function (RBF). Their method was able to successfully deal with holes
and generate water tight models. A water tight model is useful for single
object reconstruction. However, it is not desired for mapping large
environments. One of the next major advances was published in 2003 by
Ohtake et al. \cite{Ohtake2003}. In this work they created a method which
was faster than the work of Carr et al.  \cite{Carr2001} by implementing a
hierarchical approach with compactly supported basis functions. Their work
has been considered to be the state of the art for calculating an implicit
function of a noisy point set and was used by Newcombe et al.
\cite{Newcombe2010}. Volume-based methods have been able to create high
quality representations and work well for single objects and small
environments. These methods must spatially divide the environmental volume
and therefore have a high memory requirement.

\subsection{Surface-based}

One of the first interesting and adaptive surface-based methods was
published by Terzopoulos and Vasilescu in 1991 \cite{Terzopoulos1991a} and
dealt with 2.5D data such as intensity and range images. The goal of their
work was to create an adaptive mesh of an input image. The mesh was
initialized as a 2D sheet of mesh elements with virtual springs along each
edge. The stiffnesses of the virtual springs would then adjust based on the
image information at that location. The mesh was able to adapt to be more
dense in regions of higher intensity. In 1992 Terzopoulos and Vasilescu
extended their methodology to 3D data \cite{Vasilescu1992}. In this work
they used the distance between the mesh and the data to drive the vertices
to be near the surface. In this early work they needed to initialize the
mesh and control the subdivision of mesh elements to obtain a suitable
resolution. In 1993 Hoppe et al. \cite{Hoppe:1993:MO:166117.166119}
published a method to optimize to resolution of a given mesh by posing the
problem in an energy minimization framework. They minimized an energy
function which tried to adequately represent the surface in the most
concise mesh possible. One of the next advances in physical based
adaptation of meshes came in 1993 from Huang and Goldof \cite{Huang1993}.
In this work they were able to adjust the size of the mesh elements to
obtain a better resolution in areas of high frequency information using a
physical based model. In addition, it was one of the first works to
represent an object undergoing deformation. They method was able to perform
tracking on simple simulation examples. Another advancement came in 1994
Rutishauser et al. \cite{Rutishauser1994} with a method specifically
designed for incremental data. Their methodology worked with a sequential
input set of range data and used a probabilistic framework to adjust the
vertices of a mesh to the expected value given the prior observations.
Their methodology also modeled the noise of the sensor with a sensor model.
In 1994 Delingette \cite{Delingette1994} developed a methodology to
generate a simplex mesh model of structured and unstructured 3D datasets.
Elastic behavior of the mesh surface was modeled by local stabilizing
functionals. Also, they implemented an iterative refinement process to
refine the mesh in areas of high frequency information. One of the next
steps was published by Turk and Levoy in 1994 \cite{Turk1994}. Their method
allowed overlapping meshes to be ``zippered'' into a single mesh surface.
This ability is especially important for methods that generate a mesh for
each depth image of the sensor and then need to combine all registered
meshes into a single mesh. Their method is computationally expensive due to
distance calculations. An interesting work came in 1995 from Chen and
Medioni \cite{Chen1995}. They devised an adaptive mesh methodology based
on the inflation of a balloon. A mesh sphere was first initialized within
the registered range measurements of the object. Virtual inflation forces
were then used to expand the balloon until the mesh surface was a minimal
distance from the range data. This method was limited to objects which are
water tight. A major advancement came in 1999 from Bernardini et al.,
\cite{Bernardini1999a} in a method named the ball-pivoting algorithm. Their
method is a good example of an advancing front method. These types of
algorithms start with a seed mesh element and advance the boundary by
adding new mesh elements in the immediate area of the boundary which is
supported by measurements. Most advancing front algorithms differ in how it
is decided to add new mesh elements. In the work of Bernardini et al. a
virtual sphere of a user defined radius is rolled along the boundary of the
mesh and new elements are added if the ball touches another measurement.
Their methodology became popular because of its simplicity. One major
disadvantage was that the generated mesh was a fixed topology. Another
advancing front method came in 2001 from Gopi et al.
\cite{Gopi2001,Gopi2002}. Here they sampled the input dataset to obtain a new
dataset with a lower density of points in areas of lower frequency
information. This effectively gave their method an adaptive topology. Next,
a local neighborhood was computed at each data point and projected to a
plane tangent to the surface. The triangulation is then computed on this
local tangent plane. They obtained impressive results on datasets of
varying sample density and curvature. An interesting work was published in
2003 by Ivrissimtzis et al. \cite{Ivrissimtzis2003}. Here they used a
neural network model to adapt a mesh model to the data. They claimed that
their method is computationally independent of the size of the input
dataset because the dataset is only sampled by the method. There obtained
good results. In 2004 Alexa et al. published a very interesting
work to generate point set surfaces from an input dataset \cite{Alexa2004}.
They use moving least squares (MLS) to locally approximate the surface with
polynomials. The original dataset is then no longer used. Instead, they
develop tools to sample the approximated surface to any resolution desired
so that the end result is another point set of user specified resolution
lying closer to the surface than the input dataset. One drawback is they
had to develop their own methodology to render a point set. In 2005
Scheidegger et al. used the work of Alexa et al. to develop an advancing
front methodology to generate concise meshes of high accuracy. Their main
contribution was to augment an advancing front algorithm with global
information so that the triangle size could adapt gracefully to any change.
They obtained very impressive results. Most methodologies in Surface
Reconstruction had been solely concerned with object or small environment
recreation and have computational or memory requirements which do not work
well with large environments. One of the first successful methods intended
for large environments was published in 2009 by Marton et al.
\cite{Marton2009}. Their methodology was an advancing front algorithm which
worked on a point set which was sampled from the MLS surface of the
original point set. They were able to obtain impressive and near real-time
results on datasets of large environments. They also developed a method to
deal with revisited parts of the scene by determining the overlapping area
and reconstructing only the updated part of the surface mesh. While this is a
step forward, it would be better if the mesh surface converged
to the actual surface with an increasing number of measurements. To support
dynamic scenes they developed mechanisms to decouple and reconstruct the
mesh quickly. They only discussed these mechanisms in theory and had no
results of how these mechanisms work. While Surface-based Surface Reconstruction
techniques have developed impressively, several key issues arise
when applying these techniques to mapping large environments.

\section{Summary}

The fields of Robotics and Computer Vision have developed many exciting
methodologies to construct representations from a noisy input dataset. However
there is still work to be done to obtain the ideal reconstruction method. A mesh
is clearly a desirable type of representation. An ideal method should be able to
generate and maintain a mesh representation efficiently. Also, many existing
methods do not leverage the inherent structural information contained within the
depth image. There are imaging processing techniques that could be used to
answer some of the remaining problems in surface reconstruction, such as the
need for adaptive topology and the need to decide how each measurement should be
used to update the existing mesh. Henry et al. \cite{Henry2012} has already
investigated using the difference between the expected and actual measurements
to guide the decision of how to use each measurement. However, their work was
intended for surfels and needs to be extended to meshes. A method to generate a
representation is needed which is computationally and memory efficient and can
better adapt the representation to new information.


\chapter{Approach}	\label{chapter:approach}

\section{Algorithmic Design}
\label{section:algorithmic_design}

The algorithmic structure of MABDI can be seen in the system diagram shown in
Fig. \ref{fig:system}. Table \ref{tab:var} gives a description of the main
variables.

\begin{figure}[h!]%[thpb]
\centering
  \includegraphics[width=.70\textwidth]{figures/approach_mabdi_algorithm.pdf}
  \caption{MABDI system diagram}
  \label{fig:system}
\end{figure}

\begin{table}[h]
  \caption{Description of the main variables}
  \label{tab:var}
  \begin{center}
    \begin{tabular}{c|l}
    \cellcolor{white} {\bf Variable Name} & {\bf Description} \\ % adding \cellcolor{} here fixes the vertical line between the columns for some reason
    \rowcolor{LightGray}
    $D$ & Depth image from RGB-D sensor \\
    $P$ & Pose of the sensor \\
    \rowcolor{LightGray}
    $D_n$ & Parts of $D$ that are \emph{novel} \\
    $S$ & Novel surface generated from $D_n$ \\
    \rowcolor{LightGray}
    $M$ & Global mesh \\
    \end{tabular}
  \end{center}
\end{table}

The system diagram of Fig. \ref{fig:system} is a more detailed version of the
diagram seen in Fig. \ref{fig:pipeline_mabdi}. The ``Identify Novel Data''
component, shown in Fig. \ref{fig:pipeline_mabdi}, corresponds with the
Classification component, shown in blue. This Classification component is
MABDI's contribution to the state-of-art in mesh based mapping algorithms, and
is what gives MABDI the ability to make decisions about the incoming data. The
Classification component consists of two parts:
\begin{enumerate}
    \item \textit{Generate Expected Depth Image $E$} - Here we take the global
    mesh $M$, render it using computer graphics, and use the depth buffer of the
    render window to create a depth image $E$ of what we expect to see from our
    sensor. This method requires the current pose $P$ of the actual sensor
    (simulated for our experiments).
    \item \textit{Classify Depth Image $D$} - Here we classify the actual depth
    image $D$ (simulated for our experiments) by first taking the absolute
    difference between $E$ and $D$ and thresholding, as shown in the equation
    below. If the differences are small, those points are thrown away and if the
    differences are large, those points are kept as $D_n$. The idea behind this
    is, if the difference is large, the measurements are coming from a part of
    the environment that has not been seen before, i.e. novel. We found
    $threshold\mathsmaller{=}0.01$ worked well in our simulations. The
    implication of assuming all large differences signifies novel data is that
    this version of MABDI cannot handle object removal. It is worth noting that
    MABDI can be extended to handle object removal by using the sign of the
    difference between $E$ and $D$ instead of the absolute value.
    \begin{equation}
      D_n = \lvert D - E \rvert > threshold
      \label{eqn:d_e_diff}
    \end{equation}
\end{enumerate}

% Simulated for the Experiments
The system diagram in Fig. \ref{fig:system} also shows the Input and the Surface
Reconstruction components. The Input component has been simulated for our
experiments. More details of this simulation will be covered in Chapter
\ref{chapter:experimental_setup}. The Surface Reconstruction component of the
MABDI algorithm can be implemented with any viable surface reconstruction
method. Our implementation utilizes the structural information contained within
the depth image. We will discuss this in more detail in the next section.

\section{Implementation}

\subsection{Surface Reconstruction}
\label{subsection:surface_reconstruction}

The Surface Reconstruction component, as shown in Fig. \ref{fig:system}, is
responsible for creating a surface $S$ from the novel points $D_n$. The surface
$S$ is a mesh data structure that consists of a list of vertices and elements.
Vertices are points and elements define connections between vertices. Our method
outputs a triangle mesh, and so elements define the connection between three
vertices. $D_n$ is a subset of $D$ and is a list of pixel locations. For this
discussion, it will also be useful to define $D_k$ as the set of pixels in $D$
that are not pixels of $D_n$, shown in the equation below. $D_{known}$ is
labeled with ``\emph{known}'' because it represents data from the not novel or
``known'' parts of the environment. In the equation below ``$\setminus$'' is the
set difference operator.

\begin{equation}
D_{known} = D \setminus D_n
\end{equation}

Our surface reconstruction method first defines $S$ using all pixels from $D$.
We define the topology of the elements on the depth image. We can do this
because a depth image is not a set of unorganized points, but has inherent
structural information. This characteristic of the depth image allows us to
define a topology on the 2D depth image that is preserved when projected to 3D
coordinates. The topology we define can be visualized in Fig. \ref{fig:sr_t}.
Elements of the mesh are shown in light blue and pixels from $D$ are shown as
blue dots. Next we will identify elements to remove from $S$.

\begin{figure}[h]%[thpb]
\centering
  \includegraphics[width=.70\textwidth]{figures/approach_sr_topology.pdf}
  \caption{Topology defined on the depth image (not all elements are shown)}
  \label{fig:sr_t}
\end{figure}

In order to remove elements defined by points that lie on completely different
surfaces, we use an imaging technique in the form of a convolution filter. A two
dimensional, differencing convolution filter is passed over $D$. This filter has
a magnified response at points where the difference between neighboring pixels
is large. Remembering pixel values signify depth, it is assumed pixels with
large differences between themselves and their neighbor lie on different
surfaces and therefore lie on the ``boundary'' of the real surface. A large
difference is defined by thresholding on the result of the convolution. We found
$threshold\mathsmaller{=}0.01$ worked well in our simulations. (The threshold
value is unitless because the depth image is defined by the z-component of the
view coordinates, which are normalized between 0 and 1.) Pixels identified
through this thresholding are marked as $D_{boundary}$ and are defined by the
equation below where $K$ signifies the kernel of the differencing convolution
filter.

\begin{gather}
  K = \begin{bmatrix} 2 & -1 \\ -1 & 0 \end{bmatrix} \\
  D_{boundary} = (D \ast K) > threshold
\end{gather}

Elements are removed from the $S$ if they touch pixels from the sets:
\begin{itemize}
  \item $D_{known}$ - Pixels from the known parts of the environment.
  \item $D_{boundary}$ - Pixels that lie on the boundary of the actual surface.
  \item $D_{invalid}$ -  Pixels that are invalid measurements. The RGB-D sensor
  naturally has pixels that are invalid, for example, those that are out of
  range.
\end{itemize}

Let us combine the sets defined above into one set $D_{throwaway}$:
\begin{equation}
  D_{throwaway} = D_{known} \cup D_{boundary} \cup D_{invalid}
  \label{eqn:throwaway}
\end{equation}

Our method removes elements that contain pixels from the set $D_{throwaway}$.
This can be seen in Fig. \ref{fig:sr_em}. Red dots signify pixels from
$D_{throwaway}$ and elements that contain these pixels are removed from $S$. In
the final step, all pixels are projected into 3D coordinates using the
transformation matrix of the sensor. These coordinates are the vertices of $S$.

\begin{figure}[h]%[thpb]
\centering
  \includegraphics[width=.70\textwidth]{figures/approach_sr_element_removal.pdf}
  \caption{Removal of elements}
  \label{fig:sr_em}
\end{figure}

Our surface reconstruction method was chosen for its ability to be implemented
simply and run quickly. One consequence of our method is that the resulting
surface $S$ can have a large number of elements. For example, if no points are
contained in the set $D_{throwaway}$ (this can happen on the first frame), $S$
will contain over 600,000 elements. We can see this by looking at Fig.
\ref{fig:sr_t}, assuming a depth image of size 640$\times$480, and considering
the equation below.

\begin{equation}
  612,162 = ((640-1)\times2)\times(480-1)
\end{equation}

Many surface reconstruction methods have been developed to create a surface more
intelligently than our surface reconstruction method, as discussed in Chapter
\ref{chapter:related_works}. For example, the advancing front method developed
by Marton et al. \cite{Marton2009} is capable of creating surfaces with fewer
elements than our method by utilizing a robust resampling method. A capability
of the MABDI algorithm is that the method developed by Marton et al. can be used
in place of our surface reconstruction method. This characteristic of MABDI is
advantageous because MABDI does not depend on the choice of surface
reconstruction method and the method can be chosen as the state-of-the-art
changes or to suit a particular application. Also, due to our implementation's
modular software design, the entire code base would not need to be changed in
order to accomplish this. We will discuss the software design in the next
section.

\subsection{Software Design}

From a software perspective, the major difficulty of implementing the MABDI
algorithm was found to be creating both the simulated depth image $D$ and the
expected depth image $E$. In addition, managing the complexity of the data
pipeline needed to run the algorithm and the simulation of the sensor proved to
be difficult. Thankfully, Kitware, which is a leading edge developer of
open-source software, created the Visualization Toolkit (VTK)
\cite{schroeder2004visualization, sitevtk}. At the time of this writing the VTK
Github repository has over 60,000 commits and is contributed to by supporters
such as Sandia National Labs \cite{sitesandia}.

VTK is suitable for the implementation of MABDI for many reasons. Perhaps
the most important is the concept of a vtkAlgorithm (often called a Filter).
This allows a programmer to create a custom and modular processing pipeline by
defining classes that inherit vtkAlgorithm and then defining the connections
between these classes. For example, you could have a pipeline that reads an
image from a source (component 1), performs edge detection (component 2), and
then renders the image (component 3).

Using the concept of VTK filters, the individual elements of MABDI can be
succinctly defined in individual classes. With that in mind, we can see in Fig.
\ref{fig:software} the layout used in our implementation of MABDI. vtkImageData
and vtkPolyData are VTK types used to represent an image and mesh respectively.
The elements shown in blue in Fig. \ref{fig:software} are the core components of
the MABDI algorithm and are implemented as custom VTK filters. Their source code
is included in Appendix \ref{appendix:mabdi_code}. Here we will discuss all
components in detail:

\begin{sloppypar}
\begin{itemize}
    \item  \textit{Source} - Classes with the prefix Source define the
    environment that is used for the simulation and provide a mesh in the form
    of a vtkPolyData.
    \item \textit{FilterDepthImage} - Render the incoming vtkPolyData in a
    window and output the depth buffer from the window as a vtkImageData. The
    output additionally has pose information of the sensor.
    \item \textit{FilterClassifier} - Implements the true innovation of MABDI,
    i.e., takes the difference between the two incoming depth images
    (vtkImageData) and outputs a new depth image where the data that is not
    novel is marked to be thrown away.
    \item \textit{FilterDepthImageToSurface} - Performs surface reconstruction
    on the novel points. For more detail see Section
    \ref{subsection:surface_reconstruction}. The surface is output as a
    vtkPolyData.
    \item \textit{FilterWorldMesh} - Here we simply append the incoming novel
    surface to a growing global mesh that is also output as a vtkPolyData.
\end{itemize}
\end{sloppypar}

\begin{figure}[b]%[thpb]
\centering
\includegraphics[width=.75\textwidth]{figures/approach_software_diagram.pdf}
\caption{MABDI software diagram}
\label{fig:software}
\end{figure}

MABDI is implemented in Python and uses VTK. Our implementation is distributed
under the BSD license and is available on Github at the address below:

$$
https://github.com/lucasplus/MABDI
$$

At the time of this writing, it consists of over 1,400 lines. The code that
implements the MABDI algorithm itself is around 750 lines.


\chapter{Experimental Setup}	\label{chapter:experimental_setup}

MABDI was developed and tested in a completely simulated environment for several
reasons. First, all results are repeatable. Having repeatable results is
important for algorithm development because the effects of code changes in the
implementation can be directly correlated to changes in the output. This
facilitates isolation and identification of trouble spots in the code. In
addition, it is possible to test the algorithm in the most ideal environment
before adding complexity. The ability to ramp up the difficulty of the
environment in which MABDI is performing is important for making informed design
decisions. Finally, by performing the analysis in simulation we can quickly see
how the map produced by MABDI compares with the simulated environment. This
comparison is an important tool for development.

In this chapter we will give an overview of the simulation environment, discuss
how noise was generated to mimic the input of a real RGB-D sensor, and look at
the parameters chosen for the experimental runs.

\section{Simulation Overview}

For the experiments, we simulate a sensor moving in a fixed environment along a
defined path. The simulation consists of two main coordinate systems. A
coordinate system fixed to the environment called the global coordinate system
and one attached the origin of the sensor's viewing frustum. Fig.
\ref{fig:simulation_overview} shows the two coordinate systems from two
different vantage points. In the figure red, green, and blue arrows represent
the x, y, and z axis respectively.

\begin{figure}[h]%[thpb]
\centering
  \includegraphics[width=\textwidth]{figures/expsetup_simulation_overview.pdf}
  \caption{Overview of the simulation. Left: Top view. Right: Third person view. }
  \label{fig:simulation_overview}
\end{figure}

\section{Simulating a RGB-D Sensor}

\subsection{Rendering Pipeline}

In order to simulate the depth output of a RGB-D sensor, the environment is
rendered from the sensor's point of view. The rendering process produces a depth
image and this image is used as the simulated output of the sensor. Rendering is
performed by the Open Graphics Library (OpenGL). OpenGL creates a rendering
pipeline that consists of a series of transformations to project 3D global
coordinates to 2D pixel coordinates. A diagram of the rendering pipeline is
shown in Fig. \ref{fig:render_pipeline}. $T_{pcm}$ represents the pinhole
camera model and transforms geometry in the sensor's coordinate system to
homogenous coordinates. The z-component of the homogenous coordinates is what
defines the depth image. Note, the use of a pinhole camera model for simulating
RGB-D output has been validated in the localization work of Fallon
\cite{Fallon2012} and the intrinsic camera parameters of the model were chosen
to replicate the Kinect sensor \cite{sitekinectspecs}.

\begin{figure}[h]%[thpb]
\centering
  \includegraphics[width=\textwidth]{figures/expsetup_render_pipeline.pdf}
  \caption{Render pipeline: projects 3D global coordinates to 2D pixel coordinates. }
  \label{fig:render_pipeline}
\end{figure}

The pinhole camera transformation, $T_{pcm}$, creates a non-linear relationship
between values in the depth image and their corresponding location in the
sensor's coordinate system. This relationship is visualized in Fig.
\ref{fig:depth_view_to_sensor}.

\begin{figure}[h]%[thpb]
\centering
  \includegraphics[width=.70\textwidth]
    {figures/expsetup_depth_view_to_sensor.pdf}
  \caption{View coordinates to the sensor's coordinates.}
  \label{fig:depth_view_to_sensor}
\end{figure}

\subsection{Adding Noise to the Depth Image}

% Fully introduce error model, describe that it shows standard deviation as a function of distance
% Define parameters in the error model
To simulate a realistic RGB-D sensor, we add noise to the depth image $D$ with
the goal of approximating RGB-D error models from the literature. Researchers
have created error models to describe the standard deviation of measurement
error found in various RGB-D sensors. For this work, we seek to match the
well-known error model of Khoshelham \cite{Khoshelham2012} that is based on the
original Kinect. The error model is defined in Equation \ref{eqn:k_error_model}.
The equation expresses the standard deviation of error in the z-component of a
point in the sensor's coordinate system $\sigma_z$ (cm) as a function of the
value of the z-component $Z$ (m). Measurements further away from the sensor have
a larger standard deviation of error. The error model is graphed as the red line
in Fig. \ref{fig:depth_noise_error}.

\begin{equation}
  \sigma_z = 1.425\mathrm{e}{-5} \times Z^2
  \label{eqn:k_error_model}
\end{equation}

To approximate a real RGB-D sensor that matches Khoshelham's noise model, noise
is added to the depth image $D$ by sampling a normal distribution and adding the
value to each pixel. as defined in the equation below. The mean of the normal
distribution in Equation \ref{eqn:depth_noise}, $\sigma\mathsmaller{=}0.002$,
was experimentally found to provide a conservative approximation of Khoshelham's
error model.

\begin{equation}
  D_{noisy}(i,j) = D(i,j) + \mathcal{N} (\mu\mathsmaller{=}0, \sigma\mathsmaller{=}0.002)
  \label{eqn:depth_noise}
\end{equation}

In order to compare the magnitude of the standard of deviation of error used in
our experiments with that of Khoshelham's error model, we graph them on the same
plot (Fig. \ref{fig:depth_noise_error}). Each line shows how the measurement's
standard deviation of error changes as the point moves along the z axis in the
sensor's coordinate system. The standard deviation of error simulated in our
experiments is larger than that defined by Khoshelham's model for points within
the sensor's range. Therefore, our experiments are a conservative estimate of
the error found in real world RBG-D sensors.

\begin{figure}[h]%[thpb]
\centering
  \includegraphics[width=.70\textwidth]{figures/expsetup_noise_error.pdf}
  \caption{Comparison of standard deviation of the error used in the MABDI simulation and the error model from Khoshelham.}
  \label{fig:depth_noise_error}
\end{figure}

\section{Sensor Path}

All experimental runs define a helical path for the sensor to follow during the
simulation. The path is shown in Figure \ref{fig:sensor_path}. The blue line
indicates the path and the pink points indicate where the sensor stops along the
path. The path circles the objects in the environment twice. A helical path was
chosen because it returns to a part of the environment that has already been
mapped and is thus ``known'' to the algorithm. Also, because the path is a
helix and not just a circle, the sensor views the environment from a slightly
different position on each pass.

\begin{figure}[h]%[thpb]
\centering
  \includegraphics[width=.70\textwidth]{figures/expsetup_path.png}
  \caption{View of the sensor path. The blue line indicates the sensor path and the pink points indicate where the sensor stops along the path.}
  \label{fig:sensor_path}
\end{figure}

\section{Simulation Parameters}

The simulation was designed to be highly configurable and is implemented by a
class named MabdiSimulate. This class is responsible for connecting all the
components expressed in Fig. \ref{fig:software} of Chapter
\ref{chapter:approach}. MabdiSimulate is initialized with parameters that
control all aspects of the simulation. Parameters of a particular importance are
discussed in more detail here:

\begin{itemize}
    \item Environment - This parameter specifies the environment used to generate
    the simulated depth images. \textit{Table} is an environment consisting of a
    table and two cups placed on the table. The table is 1 meter tall.
    \textit{Bunnies} is an environment consisting of three bunnies that are
    around 1.5 meters tall. These bunnies are created using the Stanford Bunny
    \cite{Turk1994}, a well known data set in computer graphics.
    \item Noise - If true, adds noise to the depth image of the simulated sensor.
    \item Dynamic - If true, adds an object during the simulation. In the case
    of this analysis, a third bunny is added half-way through the simulation.
    \item Iterations - The number of times MABDI will run. This number is equal to the number of stops the sensor makes along the path because every time the sensor stops MABDI is run to update the global mesh. Figure \ref{fig:sensor_path} shows sensor stops along the sensor path.
\end{itemize}

We will be exploring three experimental runs to demonstrate the
ability of the MABDI implementation to generate valid results. Additionally, the
experimental runs will be able to show the capabilities of the MABDI algorithm
such as handling object addition in the environment.

\begin{table}[h]
  \caption{Description of the experimental runs.}
  \label{tab:run}
  \begin{footnotesize}
  \begin{center}
    \begin{tabular}{|l|c|c|c|c|}
    \hline
           & Environment & Noise   & Dynamic & Iterations \\\hline
    Run 1	 & Table       & False   & False   & 30 \\
    Run 2  & Bunnies     & True    & False   & 50 \\
    Run 3  & Bunnies     & True    & True    & 50 \\
    \hline
    \end{tabular}
  \end{center}
  \end{footnotesize}
\end{table}


\chapter{Results} \label{chapter:results}

% overview of runs
% explanation of the dashboard view
% dashboard results
% resultant mesh results

For each experimental run, a dashboard view was created that can be shown for
each iteration of the simulation. The dashboard view combines several different
views of information useful for understanding the inner workings of the MABDI
algorithm. As an example, Figure \ref{fig:run1} shows the dashboard view for the
first experimental run. For these experiments, all dashboard views follow the
same pattern as described below:

\begin{itemize}
  \item (a) - Shows the global mesh $M$ from a third-person point of view and in
  the context of the simulated environment. The multi-colored mesh is $M$. The
  mesh is multi-colored in order to show the passage of time. For example, in
  Run1, The mesh is colored yellow, light green, and dark green for iterations
  1, 2, and 3 respectively. Additional items in the view show elements of the
  simulated environment: the wire frame corresponds to the viewing frustum of
  the sensor, the light blue helical line is the path of the sensor, and the
  translucent gray mesh is the simulated environment.
  \item (b) - Same as (a) except it shows the novel surface $S$ instead of
  the global mesh $M$.
  \item (c) - Plot showing the number of elements in the global mesh $M$
  after this iteration.
  \item (d \& e) - Actual $D$ and expected $E$ depth image
  respectively.
  \item (f) - The classified depth image. Points that will be used to generate
  the novel surface $S$ are shown in black. Points to be thrown away are shown
  in white.
\end{itemize}

The dashboard views are an excellent way to visualize important aspects of
MABDI. In the next section, Section \ref{section:results1}, we will utilize key
dashboard views to look at the behavior and performance of MABDI at one particular iteration of each experimental run. In section \ref{section:results2} we will analyze the
quality and progression of the resultant global mesh from each experiment.

\section{MABDI Performance During Experiments}
\label{section:results1}

\subsection{Experiment 1}

Figure \ref{fig:run1} shows the dashboard view of the first experiment during
the third iteration. Note that \ref{fig:run1}(a) shows $M$ after the third
iteration. As stated before, $M$ is multi-colored in order to show the passage
of time. The mesh is colored yellow, light green, and dark green for iterations
1, 2, and 3 respectively. \emph{During iteration 3, $M$ is composed of only the
yellow and light green parts.}

\begin{figure}[h]%[thpb]
\centering
  \includegraphics[width=\textwidth]{figures/diagram_run1.pdf}
  \caption{Dashboard view of the first experimental run.}
  \label{fig:run1}
\end{figure}

Examining Figure \ref{fig:run1} demonstrates how the novel surface
$S$ is appended to the global mesh $M$ after each iteration of MABDI. Let's use
the figure to follow the process. It will be useful to refer to Figure
\ref{fig:system} for this section.

\begin{sloppypar} % to get rid of weird black box thing
\begin{enumerate}
  \item Input - \ref{fig:run1}(d) shows the depth image $D$ generated from the
  simulated sensor. \ref{fig:run1}(a) shows us two important aspects to consider
  about $D$. First, the pose $P$ of the sensor is shown by looking at the
  sensor's view frustum, indicated by the blue wireframe. Second, the only
  environmental information used to generate the depth image is shown in light
  gray.
  \item Generate Expected Depth Image ($E$) - \ref{fig:run1}(e) shows the
  expected depth image $E$. \ref{fig:run1}(a) also shows us two important
  aspects to consider about $E$. First, the same pose $P$ is used to create both
  $D$ and $E$ (as indicated by the blue wire frame). Second, the only
  environmental information used to create $E$ is the yellow and light green
  parts of $M$ because that is the only information $M$ contains \emph{during}
  iteration 3.
  \item Classify Depth Image ($D$) - \ref{fig:run1}(f) visualizes the
  classification process. More specifically, it shows the points as expressed in
  Equation \ref{eqn:throwaway} in white ($D_{throwaway}$). \ref{fig:run1}(f) is
  important for understanding how MABDI works because it clearly shows which
  points will be thrown away (white) and which points will be kept for
  generating the novel surface $S$ (black).
  \item Surface Reconstruction - \ref{fig:run1}(b) shows the novel surface $S$
  in the context of the simulated environment. $S$ is constructed using all the
  points colored black in \ref{fig:run1}(f).
  \item Add Novel Surface ($S$) to Global Mesh($M$) - \ref{fig:run1}(a) shows
  the novel surface $S$ appended to the global mesh $M$ in dark green.
\end{enumerate}
\end{sloppypar}

\subsection{Experiment 2}

The second experiment gives us a clear example of how the classification process
is able to identify points from the depth image $D$ that correspond to parts of
the environment that have not been seen before. In this example the global mesh
$M$ has a partial representation of the objects in the environment and when the
sensor is moved to the next pose $P$, the new perspective reveals a portion of
the object that has not been seen before. This \emph{novel portion} of the
environment, which we will be referring to, is shown by the red ellipse in Figure
\ref{fig:run2_novel_portion}.

\begin{figure}[h]%[thpb]
\centering
  \includegraphics[width=0.8\textwidth]{figures/run2_novel_portion.png}
  \caption{\emph{Novel portion} of the environment that we will be referring to
  in this section.}
  \label{fig:run2_novel_portion}
\end{figure}

Figure \ref{fig:run2} shows the dashboard view of the second experiment during
the second iteration. Using the dashboard view, we can follow how MABDI handles
the novel portion of the object step-by-step:
\begin{enumerate}
  \item \ref{fig:run2}(a) shows the global mesh $M$. The yellow portion of the
  mesh constitutes the entirety of $M$ after the first iteration. We can see the
  novel portion of the environment was not represented in $M$ after the first
  iteration due to occlusion.
  \item \ref{fig:run2}(d) shows the depth image $D$ from the new sensor pose
  $P$. We can see the novel portion can be seen by the sensor on this iteration.
  \item \ref{fig:run2}(e) shows the expected depth image $E$. During the second
  iteration $M$ consists of only the yellow portion shown in \ref{fig:run2}(a)
  consequently, $E$ does not show any points in the area corresponding to the
  novel portion of the environment.
  \item \ref{fig:run2}(f) shows the classification process successfully
  identifying points in $D$ that correspond to the novel portion as indeed
  novel. In the figure the points are highlighted by a red circle.
  \item \ref{fig:run2}(b) shows the novel surface $S$ now represents the novel
  portion of the environment.
  \item Finally, the orange mesh in \ref{fig:run2}(a) shows the novel portion of
  the environment is now represented by the global mesh $M$.
\end{enumerate}

\begin{figure}[h]%[thpb]
\centering
  \includegraphics[width=\textwidth]{figures/diagram_run2.pdf}
  \caption{Dashboard view of the second experimental run.}
  \label{fig:run2}
\end{figure}

\subsection{Experiment 3}

Experiment three shows how MABDI reacts to object addition. Figure
\ref{fig:run3} shows the dashboard view of the third experiment during the
twenty-sixth iteration. At this iteration the middle bunny is suddenly added to
the simulated environment. We can use the dashboard view to see the behavior of
MABDI to this new object:
\begin{enumerate}
  \item In \ref{fig:run3}(d) we see the depth image $D$ shows the new bunny.
  \item In \ref{fig:run3}(e) the expected depth image $E$ does not show the new
  bunny because $M$ has no representation of the new bunny.
  \item \ref{fig:run3}(f) shows the classification process successfully
  identified the points corresponding to the new bunny as novel.
  \item The novel points are used to generate the novel surface $S$ and then $S$
  is appended to $M$, shown in \ref{fig:run3}(a \& b).
  \item The addition of the new object resulted in a $S$ with a large number of
  elements for this particular iteration. \ref{fig:run3}(f) plots the resulting
  jump in the number of elements contained with $M$.
\end{enumerate}

\begin{figure}[h]%[thpb]
\centering
  \includegraphics[width=\textwidth]{figures/diagram_run3.pdf}
  \caption{Dashboard view of the third experimental run.}
  \label{fig:run3}
\end{figure}

\section{Global Mesh Results}
\label{section:results2}

\subsection{Mesh Quality}

% quality of the mesh
Figure \ref{fig:gm_3_full} shows the resultant global mesh from experiment 3. In
this section we will use this figure to make observations about the quality of
the global mesh for all three experiments.

There are gaps in the mesh that occur typically along the boundaries
of where the novel surface $S$ is appended to the global mesh $M$. This behavior
is common for Surface Reconstruction methods as those discussed in Section
\ref{section:surface_reconstruction}. Algorithms exist for merging these gaps as
a post processing step such as Turk's Zippered Polygon Meshes \cite{Turk1994}.
The aforementioned methods are typical for single object reconstruction.
Traditional mesh-based environmental mapping algorithms simply append
overlapping layers of mesh resulting in no gaps but a heavily redundant
representation with a high memory cost.

The mesh is noisy. This noisiness is due to the simplicity of our
implementation's surface reconstruction method as discussed in Section
\ref{subsection:surface_reconstruction}. Our method simply connects neighboring
points in the point cloud without additional steps such as Laplacian smoothing
\cite{Nealen2006}. Our reconstruction method was sufficient for
demonstrating the usefulness of the MABDI algorithm, but results in a mesh with
the same magnitude of noise as the sensor's simulated noise.

\begin{figure}[h]%[thpb]
\centering
  \includegraphics[width=\textwidth]{figures/run3_global_mesh.png}
  \caption{Global mesh at the end of experiment 3.}
  \label{fig:gm_3_full}
\end{figure}

\subsection{Mesh Progression}

% shape of the graphs
To appreciate the true benefit of the MABDI algorithm it is helpful to look at
how the number of elements in the global mesh $M$ progress over time. In this
section we will analyze plots showing how the number of elements in $M$ change
during the experiments. Note, the dashboard views also showed this plot. For
example, the plot of Figure \ref{fig:gm_1} is the same as Figure
\ref{fig:run1}(c), but Figure \ref{fig:gm_1} shows the plot at the completion of
the experiment.

Figure \ref{fig:gm_1} shows the resultant mesh and mesh progression for the
first experiment. The plot highlights the major difference between MABDI
and traditional mesh-based environmental mapping methods. Traditional methods
would have a plot similar to that indicated by the red arrow on the graph
because these methods have no ability to identify or remove redundant mesh
elements. Due to MABDI's algorithmic design, MABDI has the intrinsic ability to
identify points in the depth image corresponding to parts of the environment
that are already known by the global mesh $M$. MABDI then simply does not use
those points for surface reconstruction and consequently does not create
redundant mesh elements. For this reason, the number of elements in $M$ levels
off as the environment becomes more known.

\begin{figure}[h]%[thpb]
\centering
  \includegraphics[width=\textwidth]{figures/diagram_run1_gm.pdf}
  \caption{Experiment 1 global mesh results.}
  \label{fig:gm_1}
\end{figure}

Figure \ref{fig:gm_2} shows us the resultant mesh after the second experiment.
Here we can see that MABDI is reactive to the environment. In the preceding
experiment, the environment was symmetrical. In this experiment, the environment
is not symmetrical and we can see the effects by looking at the progression of
the global mesh $M$. First let us note that the sensor circles the objects twice
during the experiment and in total travels $720^{\circ}$ during the 50
iterations. We notice when the sensor gets to $90^{\circ}$ (around iteration
7) the number of elements begins to level off and then increases again as the
sensor travel to $270^{\circ}$ (around iteration 19). This behavior occurs
because the information rich perspectives of the environment occur at
$0^{\circ}$ and $180^{\circ}$. There is less for the sensor to look at when
viewing the environment from the sides. In this way, MABDI is reactive as the
sensor moves to parts of the environment that are rich in information.
Consequently, the mesh grows rapidly based on the needs of the environment.

\begin{figure}[h]%[thpb]
\centering
  \includegraphics[width=\textwidth]{figures/diagram_run2_gm.pdf}
  \caption{Experiment 2 global mesh results.}
  \label{fig:gm_2}
\end{figure}

Figure \ref{fig:gm_3} shows us the resultant mesh after the third experiment. In
this experiment the middle bunny was added during the twenty-sixth iteration.
This object addition had two effects on the global mesh. First, it created a
sudden jump in the plot as highlighted by the red circle. Second, the middle
bunny is colored blue in the resultant mesh, signifying that it was added to $M$
during a different iteration than the bunnies on the left and the right. Both of
these effects indicate that MABDI was able to successfully identify the new
bunny as novel and incorporate the bunny in to the global mesh within one
iteration.

\begin{figure}[h]%[thpb]
\centering
  \includegraphics[width=\textwidth]{figures/diagram_run3_gm.pdf}
  \caption{Experiment 3 global mesh results.}
  \label{fig:gm_3}
\end{figure}

% Figure \ref{fig:gm} shows the resultant global mesh $M$ for each of the experiments
% along with a plot of the number of elements in the mesh over iterations. These
% plots show the main contribution of MABDI because they level-off as the
% environment becomes more known as opposed to traditional reconstruction methods
% where the number of elements increases linearly over time.


\chapter{Conclusion} \label{chapter:conclusion}

The goal of MABDI is to determine data from the sensor that has not yet been
represented in the map and use this data to add to the map. MABDI does this by
leveraging the difference between what we are actually seeing and what we expect
to see. MABDI can work in conjunction with any current mesh-based surface
reconstruction algorithms, and can be thought of as a general means to provide
introspection to those types of reconstruction methods.

 The MABDI implementation was able to successfully perform in a realistic
 simulation environment. The results show how novel sensor data was
 successfully classified and used to add to the global mesh. Also, the MABDI
 algorithm runs at around 2Hz on a consumer grade laptop with an Intel i7
 processor. This performance means that it is capable of real-world
 applications.

 Currently MABDI is only designed to handle object addition, but the idea can be
 extended to handle both object addition and removal as discussed in Section
 \ref{section:algorithmic_design}. This would give the system the
 capability to handle highly dynamic environments such as a door opening and
 closing.



\appendix % __________________________________________________________ Appendix

\chapter{MABDI code}
\label{appendix:mabdi_code}
\section{FilterDepthImage.py}
\lstinputlisting[language=Python, title=FilterDepthImage.py]{code/FilterDepthImage.py}
\section{FilterClassifier.py}
\lstinputlisting[language=Python, title=FilterClassifier.py]{code/FilterClassifier.py}
\section{FilterDepthImageToSurface.py}
\lstinputlisting[language=Python, title=FilterDepthImageToSurface.py]{code/FilterDepthImageToSurface.py}
\section{FilterWorldMesh.py}
\lstinputlisting[language=Python, title=FilterWorldMesh.py]{code/FilterWorldMesh.py}


\bibliographystyle{IEEEtran}
\bibliography{bibliography}

\end{document}
