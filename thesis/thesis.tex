%\documentclass{unmeereport}
%\documentclass[final]{eeceTR} - using the final option changes layout
\documentclass[botnum, nobox]{unmeethesis}

% TODO Numbering starts on first chapter. Should be roman numerals before that, not required.

\usepackage{multirow} % for the table
\usepackage{relsize}
\usepackage{graphicx}

% \date{December 2016}
%
% \title{Mesh Addition Based on the Depth Image (MABDI)}
%
% \author{Lucas E. Chavez \\ \\
% \bf{Previous Degrees:} \\
% B.S., Mechanical Engineering \\
% New Mexico Institute of Mining and Technology, 2009\\
% }
%
% \abstract{

% }
%
% \keywords{environmental mapping, depth senor, RGB-D}

\begin{document}

\frontmatter

% Uncomment the next command if you see weird paragraph spacing:
% That is, if you see paragraphs float with lots of white space
% in between them:

% \setlength{\parskip}{0.30cm}


\title{Mesh Addition Based on the Depth Image (MABDI)}

\author{Lucas E. Chavez}

\degreesubject{B.S., Mechanical Engineering}

\degree{Master of Science \\ Mechanical Engineering}

\documenttype{Thesis}

\previousdegrees{B.S., Mechanical Engineering \\
New Mexico Institute of Mining and Technology, 2009}

\date{December, 2016}

\maketitle

\begin{dedication}
   To my parents, Albert II and Gladys, for their support,
   encouragement and the Corvette they're giving me for graduation. \\[3ex]
   ``A bird in hand is worth two in the bush''
         -- Anonymous
\end{dedication}

\begin{acknowledgments}
   \vspace{1.1in}
   I would like to thank my advisor, Professor Martin Sheen, for his support
   and some great action movies.  I would also like to thank my dog, Spot,
   who only ate my homework two or three times.  I have several other people
   I would like to thank, as well.\footnote{To my brother and sister, who
   are really cool.}
\end{acknowledgments}

\maketitleabstract %(required even though there's no abstract title anymore)

\begin{abstract}
  Many robotic applications, especially those whose goal is to aid or assist
  through human-robot interaction, utilize a rich map of the world for reasoning
  tasks such as collision detection, path planning, or object recognition. Such
  map, and the method used to produce it, must take into consideration real-world
  constraints. Most mesh-based mapping algorithms resemble a ``black box'' and do
  not provide a mechanism to close the loop and make decisions about the
  incoming information. MABDI leverages the global mesh by finding the difference
  between what we expect to see and what we are actually seeing, and using this to
  classify the incoming measurements as novel or not. This allows the surface
  reconstruction method to be run only on data that has not yet been represented in
  the global mesh. The result is an algorithm that becomes computationally
  inexpensive once the environment is known, but can also react to new objects.
\clearpage %(required for 1-page abstract)
\end{abstract}

\tableofcontents
\pagebreak

\mainmatter


\section{Introduction} \label{sec:introduction}

Many robotic applications, especially those that involve human-robot
interaction, often require a rich representation of the environment in order to
perform such behavior as path planning and obstacle avoidance. In general, a
rich representation, or map, is useful for providing situational awareness to an
autonomous agent. A map is also important for applications such as teleoperation
\cite{Kadous2006}.

The methodology to build this representation is a continuously evolving subject
in the field of robotics. The origins of the research into this problem date
back roughly 25 years \cite{Lorensen1987}. Since then the methods and the
representations themselves have continued to evolve at an impressive rate. The
main catalyst behind this growth is the advancement of sensing technologies over
the same time period. In general, sensors have continued to generate
measurements at higher rates, higher resolution, and lower cost over the years.
This has provided an amazing opportunity to build richer and more useful
representations of the environment.

In robotics, map building in an unknown environment is referred to as the
Simultaneous Localization and Mapping (SLAM) problem \cite{Thrun2002}. This
label describes the fact that a methodology which solves the SLAM problem must
simultaneously locate the robot in the environment as well as map the
environment. The focus of this work is the mapping aspect of the SLAM problem.
Fig. \ref{fig:goal} gives a visualization of the goal.

\begin{figure}[h]%[thpb]
\centering
\includegraphics[width=.5\textwidth]{figures/diagram_goal.png}
\caption{Goal is to create a map from depth images}
\label{fig:goal}
\end{figure}

There are different types of data structures that can define a map. All types have both
intrinsic characteristics that impact the algorithms that generate them and
constraints that must be considered for real-world applications. In
addition, we are concerned with rich representation types, in contrast to sparse
representation types \cite{Dissanayake2001}, because rich types have the most
use in applications such as human-robot interaction.

\begin{table}[h]
  \caption{Comparison of constraints for different map types}
  \label{tab:rep}
  \begin{footnotesize}
  \begin{center}
    \begin{tabular}{|l|c|c|c|c|c|}
    \hline
    \multirow{2}{*}{} & Supported & Computationally & Low Memory \\
     & & Inexpensive & Requirement \\\hline
    Point Clouds		& x & x & - \\
    Surfels             	& - & x & x \\
    Implicit Functions 	& x & - & - \\
    Mesh	 	& x & x & x \\
    \hline
    \end{tabular}
  \end{center}
  \end{footnotesize}
\end{table}

When considering which type of map is best for real-world applications, we must
consider the constraints imposed by each type:

\begin{itemize}
  \item Supported - Is there software, tools, research, algorithms, etc., for
  this type of map?
  \item Computationally Inexpensive - Can the algorithms run quickly on low cost
  computers (rather than specialized hardware)?
  \item Low Memory Requirement - Can the algorithms run on hardware with
  a standard amount of RAM?
\end{itemize}

Table \ref{tab:rep} compares the constraints of common map types. We can see, in
general a mesh type map satisfies real-world constraints. It has been used
extensively by the gaming and graphics communities, and so benefits from an
incredible amount of continued research and advances in hardware such as
Graphics Processing Units (GPUs).

Currently, one of the issues with mesh mapping techniques is they are generally
``black box'' methods. Meaning the data comes in from the sensor, those
measurements are turned into a mesh, and then that mesh is appended to a global
mesh. Fig. \ref{fig:pipeline} visualizes this common pipeline in black. The goal
of this work is to design an algorithm to close the loop (as visualized in red)
and allow the system to make decisions about the incoming data based on what it
already knows.

\begin{figure}[h]%[thpb]
\centering
\includegraphics[width=.5\textwidth]{figures/diagram_general_pipeline.png}
\caption{Common ``black box'' pipeline in black. The contribution of MABDI in red.}
\label{fig:pipeline}
\end{figure}

% set up what environmental mapping is, what a mesh is
% design goals of the system

\input{s_related_works}
% show many mesh based algorithms are black box
% systems capable of introspection such as kinectfusion rely on volumetric
% representation and are computationally and memory expensive

\input{s_approach}
% How Mabdi works, describe the algorithm
% How Mabdi is implemented

\section{Experimental Setup}	\label{sec:related_works}

It was decided to develop and test MABDI in a completely simulated environment
so that all results could be repeatable and also to facilitate the ability to
debug during the development process. This ability was truly invaluable as some
components of the algorithm proved to be complex from an implementation
perspective. Examples include the code to project points from the depth image to
real-world coordinates and the code for the surface reconstruction method. Being
able to debug the system by stopping the simulation at any point and inspecting
objects was nearly a necessity. In addition, we can now compare the resultant
global mesh to ground truth.

\subsection{Simulation Parameters}

The simulation was designed to be highly configurable and is implemented by a
class named MabdiSimulate. The class is initialized with parameters that control
all aspects of the simulation. Parameters of a particular importance are
discussed in more detail here:

\begin{itemize}
    \item  Environment - This parameter specifies the environment used to generate
    the simulated depth images. \textit{Table} is an environment consisting of a
    table and two cups placed on the table. The table is 1 meter tall.
    \textit{Bunnies} is an environment consisting of three bunnies who are
    around 1.5 meters tall. These bunnies are created using the Stanford Bunny
    \cite{Turk1994}, a well known data set in computer graphics.
    \item Noise - If true, adds noise to the depth image of the simulated sensor.
    \item Dynamic - If true, adds an object during the simulation. In the case
    of this analysis, a third bunny is added half-way through the simulation.
\end{itemize}

% parameters chosen the experimental runs
For this paper we will be exploring three experimental runs to demonstrate the
ability of the MABDI implementation to generate valid results. Additionally, the
experimental runs will be able to show the capabilities of the MABDI algorithm
such as handling object addition in the environment.

\begin{table}[h]
  \caption{Description of the experimental runs.}
  \label{tab:run}
  \begin{footnotesize}
  \begin{center}
    \begin{tabular}{|l|c|c|c|}
    \hline
           & Environment & Noise   & Dynamic \\\hline
    Run 1	 & Table       & False   & False       \\
    Run 2  & Bunnies     & True    & False       \\
    Run 3  & Bunnies     & True    & True       \\
    \hline
    \end{tabular}
  \end{center}
  \end{footnotesize}
\end{table}

All experimental runs define a helical path for the sensor to follow during the
simulation. The path circles the objects in the environment twice. A helical
path was chosen because it returns to a part of the environment that has been
already mapped and is thus ``known'' to the global mesh. Also, because the path
is a helix and not just a circle, the sensor views the environment from a
slightly different position on each pass.

\subsection{Analysis of Simulated Noise}

In order to realistically simulate the sensor in a real-world environment we add noise to the depth image $D$. See Fig. \ref{fig:system}. The 

As new RGB-D sensors have been developed such, as the Asus Xtion and the Kinect for Xbox One, the accuracy of the sensors has improved \cite{lachat2015first}.

Popularly accepted noise model for the kinect sensor. \cite{Khoshelham2012}

% Can add equation showing addition of noise if there is space later

% Viewpoint coordinates to real-world coordinates analysis. Viewpoint coordinates
% are obtained when a mesh is rendered into a render window, and can be
% transformed to real-world coordinates using the transformation matrix of the
% camera. Noise is added in simulation to the viewpoint coordinates. This graph
% shows the effect of that noise in real-world coordinates.

\begin{figure}[h]%[thpb]
\centering
\includegraphics[width=.5\textwidth]{figures/plot_depth.png}
\caption{Viewpoint coordinates to real world coordinates analysis.}
\label{fig:depth}
\end{figure}

% describe the simulation environment
% maybe determine speed of walker
% discuss noise

\section{Results} \label{sec:results}

\begin{figure}[h]%[thpb]
\centering
\includegraphics[width=0.48\textwidth]{figures/diagram_run123_gm.png}
\caption{Global mesh results.}
\label{fig:run1d}
\end{figure}

Results of all experimental runs. For each run: Top-left is the global mesh,
top-middle is the novel surface, top-right is the number of elements in the
global mesh, bottom-left what we actually saw, bottom-middle what we expected to
see, bottom-left threshold of the difference between expected and actual.

% Results of the runs
% Definitely talk about how adding a bunny worked well

\input{s_conclusion}
% discuss difficulty with noise
% how to expand to deal with object deletion

\section{Acknowledgment} \label{sec:results}

This work was support in part by Sandia National Laboratories under Purchase Order: 1179196 and  NSF grant OISE \#1131305.

\appendix

\section{Data}
First appendix.

\bibliographystyle{IEEEtran}
\bibliography{bibliography}

\end{document}
